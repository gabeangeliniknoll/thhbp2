% root file is THHBP2BP1.tex

\subsection{The $H\mathbb{Z}$-Bockstein spectral sequence}
Recall that there is an isomorphism of $\mathcal{A}_*$-comodules
\[
H_*(S/p\wedge THH(\B)) \cong 
	\begin{cases} 
		E(\btau_0)\otimes H_*(\B)\otimes E(\sigma \bxi_1,\sigma \bxi_2,\sigma \bxi_3)\otimes P(\sigma \btau_3) \text{ if } p=3 \\
		E(\bxi_1)\otimes H_*(\B)\otimes E(\sigma \bxi_1^2,\sigma \bxi_2^2,\sigma \bxi_3^2)\otimes P(\sigma \bxi_4) \text{ if }p=2 
	\end{cases} 
\]
where the coaction on $x\in \mathcal{A}_*$, denoted $\nu(x)$ ,is given by the restriction of the coproduct $\Delta$ of the dual Steenrod algebra to $H_*(\B/p)\subset \mathcal{A}_*$ and the remaining coactions follow from \eqref{coactsigma} along with multiplicativity of the coaction.
In this section, we compute the Bockstein spectral sequence
\begin{equation}\label{v_0Bock} 
E^1_{*,*}=THH_*(\B;\F_p)[v_0]\Rightarrow THH(\B;\mathbb{Z}_{(p)})_p.
\end{equation}
As a direct consequence the coaction on  $\sigma \btau_3$, there is a differential
\begin{equation}
d_1(\mu_3) = v_0\lambda_3. 
\end{equation}
in the $H\mathbb{Z}$-Bockstein spectral sequence \eqref{v_0Bock}.

The following lemma follows from \cite[Prop. 6.8]{May70} by translating to the $E_{\infty}$-context (cf. the proof of \cite[Lem. 3.2]{AHL}). 
\begin{lem}
	If $d_j(x)\ne 0$ in the $H\mathbb{Z}$-Bockstein spectral sequence \eqref{v_0Bock} then 
	\[
	d_{j+1}(x^p) = v_0x^{p-1}d_j(x)
	\]
	if $p>2$ or if $p=2$ and $j\geq 2$. If $p=2$ and $j=1$ then 
	\[
	d_{j+1}(x^p) = v_0x^{p-1}d_j(x)+Q^{|x|}(d_1(x))
	\]
\end{lem}

When $p=2$, we have the differential 
\[
d_1(\mu)=v_0\lambda_3.
\]
Therefore, the error term for $d_2(\mu_3^2)$ is 
\[
Q^{16}\lambda_3 = Q^{16}(\sigma\bxi_3^2) = \sigma(Q^{16}(\bxi_3^2)) = \sigma((Q^8\bxi_3)^2) = \sigma(\bxi_4^2)=0.
\]
The first equality holds by definition of $\lambda_3$, the second equality holds because $\sigma$ commutes with Dyer-Lashoff operations by \cite{Bok85}, the third equality holds by \cite{BMMS86}, and the last equality holds because $\sigma$ is a derivation \cite{AngeltveitRognes}.
\begin{cor}
	When $p=2,3$, there are differentials
	\[
	d_{i+1}(\mu^{p^i}) = v_0^{i+1}\mu_3^{p^i-1}\lambda_3.
	\]
	Consequently, there are differentials
	\[
	d_{\nu_p(k)+1}(\mu^k) \dot{=}v_0^{\nu_p(k)+1}\mu^{k-1}\lambda_3
	\]
	where $\nu_p(k)$ denotes the $p$-adic valuation of $k$.
\end{cor}
\begin{proof}
	Let $\alpha=\nu_p(k)$. We have that $k=p^\alpha j$ where $p$ does not divide $j$. So by the Leibniz rule
	\[
	d_{\alpha+1}(\mu_3^{k}) = d_{\alpha+1}((\mu_3^{p^\alpha})^j) = j\mu_3^{p^{\alpha}(j-1)}d_{\alpha+1}(\mu_3^{p^{\alpha}}) = jv_0^{\alpha+1}\mu^{p^\alpha (j-1)}\mu^{p^{\alpha}-1}\lambda_3 = jv_0^{\alpha+1}\mu^{j-1}\lambda_3.
	\]
	Since $j$ is not divisible by $p$, it is a unit mod $p$.
\end{proof}
Now recall from \eqref{Qcoeff} that $THH_*(\B;H\Q)\cong E_{\Q}(\lambda_1,\lambda_2)$. In fact the map, $THH_*(B;H\Z_{(p)})\to THH_*(B;H\Q)$ sends $\lambda_i$ to $\lambda_i$ for $i=1,2$.
Therefore, the elements $\lambda_1,\lambda_2$ are $p$-torsion free and there are no further differentials in the $H\Z$-Bockstein spectral sequence.  We rename the following classes as follows 
\begin{equation}\label{HZgens}
	\begin{array}{cc} 
		c^{(1)}_i:=\lambda_3\mu_3^{i-1},  & d_i^{(1)}:=\lambda_1c_i^{(1)}, \\
		c_i^{(2)}:=\lambda_2c_i^{(1)}, & d_i^{(2)}:=\lambda_2d^{(1)}_i.
	\end{array}
\end{equation}


Thus we have the following
\begin{cor}\label{BHZ}
There is an isomorphism of $\Z_{(p)}$-algebras
\[ THH_*(\B;H\Z_{(p)})\cong E_{\Z_{(p)}}(\lambda_1,\lambda_2)\oplus T_0\]
where $T_0$ is a torsion $\Z_{(p)}$-module defined by 
\[ T_0=\left (\Z_{(p)}\{c_i^{(k)},d_i^{(k)} \mid  i \ge 1, 1\le k\le 2\} \right )/(p^jc_i^{(k)}, p^jd_i^{(k)} \mid j=\nu_p(i)+1 ,  i \ge 1, 1\le k\le 2) \]
where the products on the elements $c_i^{(k)}$, $d_i^{(k)}$ are specified by Formula \eqref{HZgens} and by letting all other products be zero. 
\end{cor}
\begin{comment}
\gabe{Double check that we still need the remark below.}
\begin{rem}
In particular, we note that 
\[ THH_*(\B;H\Z_{(p)}) \cong E(\lambda_1)\otimes M \]
where $M$ is a free $\Z_{(p)}$ module generated by $\{1, \lambda_2\}$ tensored with a torsion $\Z_{(p)}$-module generated by $a^{(1)}_i, b_i^{(1)}$ with the same $p$-torsion as described above. We will write 
\[ THH_*(\B;H\Z_{(p)})/(\lambda_1) := M \] 
and  
\[ \lambda_1 \cdot THH_*(\B;H\Z_{(p)}) \]
for the elements in $THH_*(\B;H\Z_{(p)})$ that are $\lambda_1$-divisible, i.e. $M\{\lambda_1\}$. 
\end{rem} 
\end{comment}
\DeclareSseqGroup \tower {}{
\class(0,0)
\DoUntilOutOfBoundsThenNMore{10}{
\class(\lastx,\lasty+1)
\structline
}
}
\DeclareSseqGroup \mu {}{
\tower(0,0)
\DoUntilOutOfBoundsThenNMore{10}{
\class(\lastx+54,\lasty)
\DoUntilOutOfBoundsThenNMore{10}{
\class(\lastx,\lasty+1)
}
}
}
\DeclareSseqGroup \mudone{}{
\class(0,0)
\DoUntilOutOfBoundsThenNMore{2}{
    \class(\lastx,\lasty+1)
     \structline
}
\class(1,0)
\DoUntilOutOfBounds{
    \class(\lastx,\lasty+1)
     \structline
    \d[blue]1(\lastx,\lasty-1)
}
}
\DeclareSseqGroup \mudtwo{}{
\class(0,0)
\DoUntilOutOfBoundsThenNMore{2}{
    \class(\lastx,\lasty+1)
    \structline
}
\class(1,0)
\DoUntilOutOfBounds{
    \class(\lastx,\lasty+1)
     \structline
    \d[blue]2(\lastx,\lasty-1)
}
}
\tiny
\begin{sseqdata}[ name = BSSv_0,classes=fill, xscale = .2, yscale=.5, title = { $E_{\page}$-page of $v_0$-Bockstein Spectral Sequence at $p=2$}, Adams grading, x tick step = 2,x range = {0}{58}, y range = {0}{5} ]
\tower
%tower on  \lambda_1
\tower(3,0)

%tower on  \lambda_2
\tower(7,0)

%towers on  \lambda_1\lambda_2\mu^k
\tower(10,0)

%differential d_1(p^k\mu)=p^{k+1}\lambda_3
\mudone(15,0)

%differential d_1(p^k\lambda_1mu)=p^{k+1}\lambda_1\lambda_3
\mudone(18,0)

%differential d_1(p^k\lambda_2mu)=p^{k+1}\lambda_2\lambda_3
\mudone(22,0)

%differential d_1(p^k\lambda_1\lambda_2mu)=p^{k+1}\lambda_1\lambda_2\lambda_3
\mudone(25,0)

%differential d_2(p^k\mu^2)=p^{k+2}\lambda_3\mu_2
\mudtwo(31,0)

%differential d_2(p^k\lambda_1\mu^2)=p^{k+2}\lambda_1\lambda_3\mu_2
\mudtwo(34,0)

%differential d_2(p^k\lambda_2\mu^2)=p^{k+2}\lambda_2\lambda_3\mu_2
\mudtwo(38,0)

%differential d_2(p^k\lambda_1\lambda_2\mu^2)=p^{k+2}\lambda_1\lambda_2\lambda_3\mu_2
\mudtwo(41,0)

%differential d_1(p^k\mu^3)=p^{k+1}\lambda_3\mu_2^2
\mudone(47,0)

%differential d_1(p^k\lambda_1\mu^3)=p^{k+1}\lambda_1\lambda_3\mu_2^2
\mudone(50,0)

%differential d_1(p^k\lambda_2\mu^3)=p^{k+1}\lambda_2\lambda_3\mu_2^2
\mudone(54,0)

%differential d_1(p^k\lambda_1\lambda_2\mu^3)=p^{k+1}\lambda_1\lambda_2\lambda_3\mu_2^2
\mudone(57,0)

\begin{comment}
%differential d_1(p^kmu^2)=p^{k+1}\lambda_3\mu
\mudone(107,0)

%differential d_1(p?k\lambda_1\lambda_3mu)=p?{k+1}\lambda_1\mu?2
\mudone(112,0)

%differential d_1(p?k\lambda_2mu?2)=p?{k+1}\lambda_2\lambda_3\mu
\mudone(124,0)

%differential d_1(p?k\lambda_1\lambda_2\mu?2)=p?{k+1}\lambda_1\lambda_2\lambda_3\mu
\mudone(129,0)



%differential d_2(p?k\lamda_1\mu?3)=p?{k+2}\lambda_1\lambda_3\mu_2
\mudtwo(166,0)

%differential d_2(p?k\lamda_2\mu?3)=p?{k+2}\lambda_2\lambda_3\mu_2
\mudtwo(178,0)

%differential d_2(p?k\lambda_1\lamda_2\mu?3)=p?{k+2}\lambda_1\lambda_2\lambda_3\mu_2
\mudtwo(183,0)

%differential d_1(p?kmu?4)=p?{k+1}\lambda_3\mu?3
\mudone(215,0)

%differential d_1(p?k\lambda_1mu?4)=p?{k+1}\lambda_1\lambda_3\mu?3
\mudone(220,0)

%differential d_1(p?k\lambda_2mu?4)=p?{k+1}\lambda_2\lambda_3\mu?3
\mudone(232,0)

%differential d_1(p?k\lambda_1\lambda_2mu?4)=p?{k+1}\lambda_1\lambda_2\\lambda_3mu?3
\mudone(237,0)

%differential d_1(p?kmu?5)=p?{k+1}\lambda_3mu?4
\mudone(269,0)

%differential d_1(p?k\lambda_1mu?5)=p?{k+1}\lambda_1\lambda_3mu?4
\mudone(274,0)

%differential d_1(p?k\lambda_2mu?5)=p?{k+1}\lambda_2\lambda_3mu?4
\mudone(286,0)

%differential d_1(p?k\lambda_1\lambda_2mu?5)=p?{k+1}\lambda_1\lambda_2\lambda_3mu?4
\mudone(291,0)

%differential d_2(p?k\mu?6)=p?{k+2}\lambda_3\mu?5
\mudtwo(323,0)
\end{comment}
\end{sseqdata}
\printpage[ name = BSSv_0, page = 1] 


\printpage[ name = BSSv_0, page = 2]

\printpage[ name = BSSv_0, page = 3] 

\normalsize 


