% root file is THHBP2BP1.tex

\section{Preliminary results}\label{sec prelim}
The homology of topological Hochschild homology of $\B$ is a straightforward application of results of \cites{BMMS86,Bok85,AngeltveitRognes} and it appears in \cite[Thm. 5.12]{AngeltveitRognes}. Recall that there is an isomorphism
\[H_*(\B)\cong \begin{cases} P(\bxi_1,\bxi_2,\ldots )\otimes E(\btau_{3},\btau_4,\ldots) \text{ if } p \ge 3 \\ P(\bxi_1^2,\bxi_2^2,\bxi_3^2,\bxi_4,\ldots) \text{ if } p=2 \end{cases}\]
of $\A_*$-comodules. Then by \cite[Thm. 5.12]{AngeltveitRognes} there is an isomorphism
\begin{equation}\label{HTHHB} H_*(\THH(\B))\cong \begin{cases} H_*\B \otimes E(\sig{\bxi_1},\sig{\bxi_2},\sig{\bxi_3})\otimes P(\sig{\btau_3})\text{ if } p = 3 \\ H_*\B\otimes E(\sig{\bxi^2_1},\sig{\bxi^2_2},\sig{\bxi^2_3})\otimes P(\sig{\bxi_4}) \text{ if }p=2\end{cases}\end{equation}
of primitively generated $H_*\B$-Hopf algebras. We also note the coaction on $H_*\THH(\B)$ as a comodule over $\A_*$ computed in \cite[Thm. 5.12]{AngeltveitRognes}
\begin{equation}\label{coactp>2}\nu ( \sig{\btau_m}) = 1\otimes \sigma \btau_m + \btau_0\otimes\sig{\bxi_m}\end{equation}
at $p = 3$ and 
\begin{equation}\label{coactp=2} \nu(\sig{\bxi_{m+1}})=1\otimes\sig{\bxi_{m+1}}+\bxi_1\otimes\sigma\bxi_2^2.\end{equation}
at $p=2$. These both follow from the formula
\begin{equation}\label{coactsigma} \nu \circ \sigma = (1\otimes \sigma)\circ \nu \end{equation}
in \cite[Eq. 5.11]{AngeltveitRognes} and the well known $\A_*$-coaction on $H_*\B$. By the same argument, $\sigma \xi_i$ is primitive at $p=3$ and $\sigma \xi_i^2$ is primitive at $p=2$ for $i=1,2,3$.
\begin{comment}
\subsection{Multiplicativity of Bockstein spectral sequences}
As described in our outline, the first goal will be to compute the Bockstein spectral sequences
\begin{align}
	\label{v_0BSS}\THH_*(\B;\F_p)[v_0]&\implies \THH_*(\B;\Z_{(p)})^{\wedge}_p\\
	\label{v_1BSS}\THH_*(\B;\F_p)[v_1]&\implies \THH_*(\B;k(1))\\
	\label{v_2BSS}\THH_*(\B;\F_p)[v_2]&\implies \THH_*(\B;k(2)).
\end{align}
We will first argue that these Bockstein spectral sequences are multiplicative when $p=3$. Recall that $THH(\B;M)\simeq THH(\B)\wedge_BM$. 
Consider the Adams spectral sequences
\begin{equation}\label{ASS} E_2^{*,*}=\Ext_{\A}^{**}(\F_p, H_*(THH(\B)\wedge_{\B}k(i)))\Rightarrow \pi_*THH(\B;k(i))_p \end{equation}
where $k(0)=\mathbb{Z}_{(p)}$ and $k(1)=\ell/p$ and $k(2)$ is the connective cover of $K(2)$. 
By \eqref{HTHHB}, we know that $H_*THH(\B)$ is free over $H_*(\B)$. Since $\B$ is a commutative ring spectrum, there is a splitting $THH(\B;M)\simeq \B\vee \overline{THH}(\B;M)$ for any $\B$-module and consequently the input becomes 
\[\Ext_{\A}^{**}(\F_p, H_*(\overline{THH}(\B))\otimes H_*(k(i)))\]
which is isomorphic to 
\[\Ext_{E(\tau_i)_*}^{**}(\F_p, H_*(\overline{THH}(\B))).\]
By \eqref{coactp>2}, the $\tau_i$ coaction on $H_*(\overline{THH}(\B))$ is trivial and therefore the input becomes 
\[\Ext_{E(\tau_i)_*}^{**}(\F_p,\F_p )\otimes H_*(\overline{THH}(\B)).\]
and $\Ext_{E(\tau_i)_*}^{**}(\F_p,\F_p )\cong P(v_i)$ for $0\le i\le 3$. Since the spectra $k(i)$ are $A_{\infty}$-ring spectra  and $\B$ is an $E_{\infty}$-ring spectrum, $THH(\B;k(i))$ is ring spectrum. Thus, the Adams spectral sequence is multiplicative. Since the Adams spectral sequence \eqref{ASS} is equivalent to the corresponding Bockstein spectral sequence, each of these Bockstein spectral sequences is multiplicative. 

We will also consider Bockstein spectral sequences 
\begin{align}
	\label{v_0v_1BSS}\THH_*(\B;\Z_{(p)})_p[v_1]&\implies \THH_*(\B;\tBP{1})\\
	\label{v_2v_1BSS}\THH_*(\B;k(2))[v_1]&\implies \THH_*(\B;\B/p)\\
	\label{v_1v_2BSS}\THH_*(\B;k(1))[v_2]&\implies \THH_*(\B;\B/p)\\
	\label{v_0v_2BSS}\THH_*(\B;\Z_{(p)})[v_2]&\implies \THH_*(\B;\B/v_1)\\
	\label{v_1v_0BSS}\THH_*(\B;k(1))[v_0]&\implies \THH_*(\B;\tBP{1})\\
	\label{v_2v_0BSS}\THH_*(\B;k(2))[v_0]&\implies \THH_*(\B;\B/v_1)
\end{align}
each of which is also multiplicative except the last two by using an Adams spectral sequence in an appropriate module category and choosing a minimal resolution to get an equality of spectral sequences at the $E_1$-page. 
\gabe{I need to verify this claim still. This is what AHL do.}
Finally, Bockstein spectral sequences 
\begin{align}
	\label{v_0v_1v_2BSS}\THH_*(\B;\tBP{1})[v_2]&\implies \THH_*(\B)\\
	\label{v_0v_2v_1BSS}\THH_*(\B;\B/v_1)[v_1]&\implies \THH_*(\B)\\
	\label{v_1v_2v_0BSS}\THH_*(\B;\B/p)[v_0]&\implies \THH_*(\B).
\end{align}
are also multiplicative except the third one by the same kind of argument as above. 
\gabe{At the moment, this is written as though we will compute $THH_*(\B)$ and it will need to be adjusted if we decide to finish the paper at $S/p_*THH(\B)$.}
\end{comment}
\subsection{THH of $\B$ modulo $(p,v_1,v_2)$}
We now compute %topological Hochschild homology $\B$ modulo $(p,v_1,v_2)$. This amounts to a computation of 
\[\THH_*(\B;H\F_p).\] 
By \cite[Lem. 4.1]{AngeltveitRognes}, it suffices to compute the sub-algebra of co-mododule primitives in $H_*(\THH(\B;H\F_p))$ since $\THH(\B;H\F_p)$ is an $H\F_p$-algebra. Since $\B$ and $H\mathbb{F}_p$ are commutative ring spectra there is a weak equivalence of commutative ring spectra
\[ \THH(\B;H\F_p)\simeq \THH(\B)\wedge_{\B} H\mathbb{F}_p. \] 
Since $H_*(\THH(\B))$ is free over $H_*\B$ by \eqref{HTHHB}, the Eilenberg-Moore spectral sequence and \cite[Cor. 5.13]{AngeltveitRognes} immediately implies
\begin{equation}\label{eqn:HTHH(R;F_p)}
H_*(\THH(\B;H\F_p))\cong \begin{cases} \A_* \otimes E(\sigma \bxi_1, \sigma \bxi_2, \sigma \bxi_3)\otimes P(\sigma \btau_3) \text{ if } p=3 \\  \A_* \otimes E(\sigma \bxi_1^2, \sigma \bxi_2^2, \sigma \bxi_3^2)\otimes P(\sigma \bxi_3) \text{ if } p=2. \end{cases}
\end{equation}
The $\A_*$ coaction on elements in $\A_*$ is given by the coproduct and the remaining coactions are determined by the formula \eqref{coactsigma} and are therefore the same as in \eqref{coactp>2} and \eqref{coactp=2}. We write $\lambda_i=\sigma\bxi_i$ at $p=3$ and $\lambda_i=\sigma \bxi_i^2$ are $p=2$. We also define 
\[ \mu_3=\begin{cases} \sigma \btau_3 -\btau_0\otimes \sigma \bxi_3 \text{ if } p=3 \\ \sigma \bxi_4-\bxi_1\otimes \sigma \bxi^2_3 \text{ if } p=2 \end{cases}. \]
Then it is clear that the algebra of comodule primitives in $H_*(\THH(\B ;H\mathbb{F}_p))$ is generated by $\mu_3$ and $\lambda_i$ for $1\le i\le 3$. We therefore produce the following isomorphism of graded $\mathbb{F}_p$-algebras
\begin{equation}\label{eqn:THH(R;F_p)}
\THH_*(\B;H\F_p)\cong E(\lambda_1, \lambda_2, \lambda_3)\otimes P(\mu_3).
\end{equation}
The degrees of the algebra generators are $|\lambda_i|=2p^i-1$ for $1\le i\le 3$ and $|\mu_3| = 2p^3$. 

\subsection{Rational homology}

Next, we compute the rational homology of $\THH(\B)$ to locate the torsion free component of $\THH_*(\B)$. Towards this end, we will use the $H\Q$-based B\"okstedt spectral sequence. This is a spectral sequence of the form 
\[
E_2^{**}=\HH_*^{\Q}(H\Q_*\B)\implies H\Q_*\THH(\B).
\]
Recall that the rational homology of $\B$ is  
\[
H\Q_*\B\cong P_\Q (v_1,v_2).
\]
Thus the $E_2$-term of the B\"okstedt spectral sequence is 
\[
P_\Q(v_1, v_2)\otimes E_\Q( \sigma v_1,\sigma v_2)
\]
and the bidegree of $\sigma v_i$ is $(1,2(p^i-1))$. Note that $\B$ is a commutative ring spectrum, so by \cite[Prop. 4.3]{AngeltveitRognes} the B\"okstedt spectral sequence is multiplicative. All the algebra generators are in B\"okstedt filtration $0$ and $1$ and the $d^2$ differential shifts B\"okstedt filtration by two, so there is no room for differentials. Thus, the $E_2$-term is isomorphic to the $E_\infty$-term as graded $\Q$-algebras. There are also no hidden extensions. %Why? Probably it is obvious, but we should include an argument. qx
Thus, there is an isomorphism of graded $\Q$-algebras
\[
\THH_*(\B)\otimes \Q\cong P_\Q(v_1,v_2)\otimes_\Q E_\Q(\sigma v_1,\sigma v_2)
\]
where $|\sigma v_i|=2p^i-1$. By the same method as \cite[Thm. 1.1]{Rog19}, one can prove that 
\[ \sigma v_1=p\lambda_1 \]
\[\sigma v_2=p\lambda_2-v_1^p\lambda_1-v_1^{p}\sigma v_1.\]
Consequently, up to a change of basis,
\begin{equation}\label{Qcoeff} THH_*(\B;H\Q)\cong E_{\Q}(\lambda_1,\lambda_2). \end{equation}
Also, we conclude that 
\[ L_0 \THH(\B)\simeq L_0\B\wedge \Sigma^{2p-1}L_0\B\vee \Sigma^{2p^2-1}L_0\B\vee\Sigma^{2p^2+2p-2}L_0\B\]
where $L_0=L_{H\mathbb{Q}}$, since $L_0$ is a smashing localization and $L_0S=H\mathbb{Q}$. 

