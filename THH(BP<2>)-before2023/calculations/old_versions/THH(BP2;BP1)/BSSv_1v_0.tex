%\documentclass[12pt]{amsart}
%\usepackage[margin=0in,landscape]{geometry}
%\usepackage{comment}
%\usepackage{spectralsequences}
%\begin{document}

\DeclareSseqGroup \tower {}{
\class(0,0)
\DoUntilOutOfBoundsThenNMore{10}{
\class(\lastx,\lasty+1)
\structline
}
}

\DeclareSseqGroup \towerfour {}{
\tower
\Do{3}{
    \class(\lastx+2,\lasty+1)
    \structline
    \DoUntilOutOfBounds{
            \class(\lastx,\lasty+1)
\structline
    } 
}
}

\DeclareSseqGroup \towereight {}{
\tower
\Do{7}{
    \class(\lastx+2,\lasty+1)
    \structline
    \DoUntilOutOfBounds{
            \class(\lastx,\lasty+1)
\structline
    } 
}
}

\DeclareSseqGroup \towersixteen {}{
\tower
\Do{15}{
    \class(\lastx+2,\lasty+1)
    \structline
    \DoUntilOutOfBounds{
            \class(\lastx,\lasty+1)
	\structline
    } 
}
}

\begin{sseqdata}[ name = BSSv_1v_0,classes=fill, xscale = .21, yscale=.5, title = { $v_1$-torsion in the $E_1$-page of the of $v_0$-Bockstein Spectral Sequence for $0\le x\le 40$ modulo $\lambda_1$}, Adams grading, y tick step = 1, x tick step = 2,x range = {0}{40}, y range = {0}{10} ] \label{fig BSSv1v0}
\towerfour(7,0)
\towereight(15,0)
\towerfour(22,0)
\towersixteen(23,0)
\towereight(38,0)
\towerfour(39,0)

%\begin{comment}
\towerfour(10,0)
\towereight(18,0)
\towerfour(25,0)
\towersixteen(26,0)
\towereight(41,0)
\towerfour(42,0)
%\end{comment}

\end{sseqdata}
\printpage[ name = BSSv_1v_0, page = 1] 


%\end{document}
