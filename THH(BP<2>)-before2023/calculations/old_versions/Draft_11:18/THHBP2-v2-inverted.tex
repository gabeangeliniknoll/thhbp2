% root file is THHBP2.tex

\section{The $v_2$-inverted homotopy of $\THH(BP\langle 2 \rangle)$}

In this section, we briefly describe the calculation of $\THH(BP\langle 2 \rangle;E(2))$ based on recent work of Ausoni-Richter \cite{qx}. In \cite{qx}, they assume that an $E_{\infty}$-model for $E(2)$ exists, so for this section we will also make this assumption. There is a map 
\[ \THH(\B;E(2))\to \THH(E(2)) \]
induced by the localization map 
\[\B\to v_2^{-1}\B\simeq E(2).\]
We first recall the main theorem of \cite{qx}
\begin{thm}[Ausoni-Richter \cite{qx}]
There is an equivalence 
\[ \THH(E(2)) \simeq E(2) \vee \Sigma^{2p-1}L_1E(2) \vee \Sigma ^{2p^2-1}L_0E(2) \vee \Sigma^{2p^2+2p-2} L_0E(2). \]
\end{thm}
Since there is an equivalence $THH(\B)\wedge_{\B}E(2)\simeq THH(E(2))$, we get the following corollay. 
\begin{cor}
There is an equivalence
\[ THH(\B;E(2))\simeq E(2) \vee \Sigma^{2p-1}L_1E(2) \vee  \Sigma ^{2p^2-1}L_0E(2) \vee \Sigma^{2p^2+2p-2}L_0(E(2)\]
\end{cor}
Note that this is consistent with our computation of $L_0THH(\B)$, our computation of $THH_*(\B;E(2)/(p,v_1))$, and out computation of 
\[ THH_*(\B;\tBP{1})\]
which we computed independently and therefore did not depend on $E(2)$ being a commutative ring spectrum. 

Note that $\pi_*(v_1^{-1}S/p\wedge E(2))\cong P(v_1^{\pm 1},v_2^{\pm 1})$. In particular, this also implies the following. 
\begin{cor}
There is an equivalence
\[ THH(\B;E(2)/p)\simeq E(2)/p \vee \Sigma^{2p-1}(L_1E(2))/p\]
and consequently, an equivalence 
\[ v_1^{-1}S/p\wedge THH(\B;E(2))\simeq v_1^{-1}E(2)/p\vee \Sigma^{2p-1}v_1^{-1}(L_1E(2))/p\]
and therefore the free part of 
$THH_*(\B;\B/p)$ as a $P(v_1,v_2)$-module is generated by $1$ and $\lambda_1$. 
\end{cor}