% root file is THHBP2.tex

\section{The $v_1$-Bockstein spectral sequence}

In this section we will begin our analysis of the $v_1$-Bockstein spectral sequence \eqref{v_1BSS} for computing the homotopy of $\THH(\B;k(1))$. We will take a similar approach to the previous section.
To start, we need to compute $K(1)_*\tBP{2}$. This requires determining $\eta_R(v_{2+n})$ in $K(1)_*BP$ modulo the ideal generated by $(\eta_R(v_3), \ldots, \eta_R(v_{1+n}))$. We will need the following. 

\begin{lem}\label{lem:rightunit}{\cite[Lemma A.2.2.5]{greenbook}} Let $v_n$ denote the Araki generators. Then there is the following equality in $BP_*BP$
\[
\sum_{i,j\geq 0}\hspace{-5pt}\mbox{}^F\: t_i\eta_R(v_j)^{p^i}=\sum_{i,j\geq 0}\hspace{-5pt}\mbox{}^F\:v_it_j^{p^i}
\]
\end{lem}

In our context, the distinction between Hazewinkel generators and Araki generators is unimportant, as the two sets of generators coincide modulo $p$. In $K(1)_*BP$, we have killed all $v_i$'s except $v_1$, which gives us the following equation
\[
\sum_{i,j\geq 0}\hspace{-5pt}\mbox{}^F\: t_i\eta_R(v_j)^{p^i}=\sum_{k\geq 0}\hspace{-4pt}\mbox{}^F\:v_1t_k^{p}
\]
Note that the following degrees of the terms:
\begin{align*}
	|v_1t_j^p|&= 2(p^{j+1}-1)\\
	|t_i\eta_R(v_j)^{p^i}|&= 2(p^{i+j}-1)
\end{align*}
Since we are interested in the term $\eta_R(v_{2+n})$, we collect all the terms on the left of degree $2(p^{2+n}-1)$. Thus we are summing over the ordered pairs $(i,j)$ such that $i+j=2+n$. Since we only care about $\eta_R(v_{2+n})$ modulo $\eta_R(v_3), \ldots, \eta_R(v_{1+n})$ we only need to collect the terms where $j=1, 2$, or $2+n$. This shows that 
\[
t_{1+n}\eta_R(v_1)^{p^{n+1}}+t_n\eta_R(v_2)^{p^n}+\eta_R(v_{n+2})=v_1t_{n+1}^p
\]
The value of $\eta_R$ on $v_1$ and $v_2$ can also be computed by Lemma \ref{lem:rightunit}. One obtains, in $K(1)_*BP$, the following
\begin{align*}
\eta_R(v_1)&=v_1\\
\eta_R(v_2)&=v_1t_1^p-t_1v_1^p.
\end{align*}
Combining these observations, we obtain

\begin{lem}
	In $K(1)_*BP$, the following congruence is satisfied
	\[
	\eta_R(v_{2+n})\equiv v_1t_{n+1}^p-v_1^{p^n}t_1^{p^{n+1}}t_n+v_1^{p^{n+1}}(t_1^{p^n}t_n-t_{n+1})  \mod(\eta_R(v_3), \ldots, \eta_R(v_{1+n}))
	\]
	for $n\geq 1$.
\end{lem}

Consequently, we have the following corollary. 

\begin{cor}
	There is an isomorphism of $K(1)_*$-algebras
	\[
	K(1)_*\B\cong K(1)_*BP/(v_1t_{n+1}^p-v_1^{p^n}t_1^{p^{n+1}}t_n+v_1^{p^{n+1}}(t_1^{p^n}t_n-t_{n+1}) \mid n\geq 1)
	\]
\end{cor}

Define elements
\[ u_n:=v_1^{\frac{p^n-1}{p-1}}t_n.\] 
These elements are in degree 0, and therefore there is an isomorphism
\[
K(1)_*\B\cong_{K(1)_*}K(1)_*\otimes_{\F_p}K(1)_0\B.
\]
The calculations above imply the following corollary.

\begin{cor}
	There is an isomorphism of $\F_p$-algebras
	\[
	K(1)_0 R\cong\F_p[u_i\mid i\geq 1]/(u_{n+1}^p-u_1^{p^{n+1}}u_n+u_1^{p^n}u_n-u_{n+1}\mid n\geq 1).
	\]
\end{cor}

Our goal is to use this and the $K(1)$-based B\"okstedt spectral sequence to compute the $K(1)$-homology of $\THH(R)$. This is a spectral sequence of the form 
\[
E^2_{s,t}= \HH^{K(1)_*}_s(K(1)_*R)\implies K(1)_{s+t}\THH(R).
\]
The above considerations tell us that the $E^2$-page is 
\[
E^2\cong K(1)_*\otimes \HH^{\F_p}(K(1)_0R). 
\]

The following will be useful for our calculation.

\begin{lem}[\cite{MilneLEC}]\label{lem:etale}
	Let $V = \Spec(A)$ be a nonsingular affine variety over a field $k$. Let $W$ be the subvariety of $V\times \mathbb{A}^n$ defined by equations
	\[
	g_i(Y_1, \ldots, Y_n)=0, \, g_i\in A[Y_1, \ldots, Y_n],\, i=1,\ldots , n.
	\]
	Then the projection map $W\to V$ is \'etale at a point $(P;b_1, \ldots, b_n)$ of $W$ if and only if the Jacobian matrix $\begin{pmatrix}
		\frac{\partial g_i}{\partial Y_j}
	\end{pmatrix}$ is a nonsingular matrix at $(P; b_1, \ldots , b_n)$.
\end{lem} 

\begin{thm}[\'Etale Descent, \cite{WeibelGeller}]\label{etaledescent}
	Let $A\hookrightarrow B$ be an \'etale extension of commutative $k$-algebras. Then there is an isomorphism
	\[
	\HH_*(B)\cong \HH_*(A)\otimes_A B
	\]
\end{thm}

\begin{ex}
	Consider the subalgebra
	\[
	\F_p[u_1,u_2]/(u_2^p-u_1^{p^2+1}+u_1^{p+1}-u_2=f_1).
	\]
	We will regard this as a $\F_p[u_1]$-algebra. The partial derivative $\partial_{u_2}f_1$ is $-1\pmod{p}$, and therefore a unit at every point. Then Lemma \ref{lem:etale} tells us that this algebra is then \'etale over $\F_p[u_1]$.
\end{ex} 

By the same argument given above, we claim that there are a sequence of sub-algebras $A_n$ of 
	\[
	K(1)_0 R\cong\F_p[u_i\mid i\geq 1]/(u_{n+1}^p-u_1^{p^{n+1}}u_n+u_1^{p^n}u_n-u_{n+1}\mid n\geq 1)=:A.
	\]
such that each map $A_i\hookrightarrow A_{i+1}$ is an \'etale extension. Here 
\[A_n:=\F_p[u_1,u_2,\ldots u_n]/(u_{k+1}^p-u_1^{p^{k+1}}u_k+u_1^{p^k}u_k-u_{k+1} =f_{k} \mid 1\le k\le n)\]
and the partial derivative $\partial_{u_k}f_{k}=-1\pmod{p}$ for all $1<k\le n$ and therefore a unit at each point. The claim then follows by Lemma \ref{lem:etale}.

By the \'etale base change formula for Hochschild homology in Theorem \ref{etaledescent}, there is an isomorphism 
\[ \HH_*^{\F_p}(A_{i+1})\cong \HH_*^{\F_p}(A_i)\otimes_{A_i}A_{i+1}\]
and since the functors $HH_*(-)$ and $ \HH_*^{\F_p}(A_1)\otimes_{A_1}(-)$ commute with filtered colimits of $\F_p$-algebras, there are isomorphisms 
	\[ 
	\begin{array}{rcl} 
		\HH_*^{\F_p}(A) & \cong &\HH_*^{\F_p}(\colim A_n) \\
				         & \cong & \colim  \HH_*^{\F_p}(A_n) \\
				         & \cong & \colim \HH_*^{\F_p}(A_1)\otimes_{A_1}A_n \\
				         & \cong & \HH_*^{\F_p}(A_1)\otimes_{A_1}A. \\
	\end{array}
	\]
Consequently,
\[ \HH_*(K(1)_*\B)\cong K(1)_*\otimes E(\sigma t_1)\otimes K_0(R) \]
and therefore, since $\sigma t_1\dot{=}\lambda_1$, 
\[ K(1)_*\THH(\B)\cong K(1)_*\B\otimes E(\lambda_1) \]
and 
\[ \THH_*(\B;K(1))\cong K(1)_*\otimes E(\lambda_1). \]
In other words, 
\[ THH_*(\B;k(1)) \cong F\oplus T\]
where $F$ is a free $P(v_1)$-module generated by $1$ and $\lambda_1$ and $T$ is a torsion $P(v_1)$-module. 
\begin{comment}	
We therefore have the input needed to compute the Eilenberg-Moore spectral sequence,
\[
\Tor^{K(1)_*R}(K(1)_*K(1), K(1)_*\THH(R))\implies K(1)_*\THH(R;K(1))
\]
From the previous computation, the $E^2$-term is concentrated in $\Tor_0$ and is 
\[
K(1)_*K(1)\otimes E(\lambda_1).
\]		         
Thus, every class besides $1$ and $\lambda_1$ is $v_1$-torsion in $\THH_*(R;k(1))$. Since $\THH(R;K(1))$ is a $K(1)$-module, this implies that 
\[
\THH(R;K(1))\simeq K(1)\vee \Sigma^{2p-1}K(1).
\]
\end{comment}
In summary, we have proven the following theorem.
\begin{thm}\label{thm:K(1)coeff}\mbox{}
	\begin{enumerate}
		\item The $K(1)$-homology of $\THH(R;K(1))$ is $K(1)_*K(1)\otimes E(\lambda_1)$
		\item Their is a weak equivalence
		\[ K(1)\vee \Sigma^{2p-1}K(1)\simeq \THH(R;K(1)).\]
		\item The only $v_1$-torsion free classes in $\THH(R;k(1))$ are $v_1^k$ and $\lambda_1v_1^k$ for $k\ge0$.
	\end{enumerate}
\end{thm}

\subsection{Differentials in the $v_1$-BSS}

We now analyze the $v_1$-BSS \eqref{v_1BSS}. Recall that this spectral sequence is of the form 
\[
\THH(\B;\F_p)[v_1]\implies \THH(B;k(1)).
\]
Thus the $E_1$-page is 
\begin{equation}\label{v_1bockinput}
E(\lambda_1, \lambda_2, \lambda_3)\otimes P(\mu_3)\otimes P(v_1). 
\end{equation}
Since the $\lambda_i$ are all in odd total degree and since $v_1^k$ are known to be $v_1$-torsion free for all $k$, the $\lambda_i$ are all permanent cycles. If $\mu_3$ were a permanent cycle, then by multiplicativity, the spectral sequence would collapse. This would contradict Theorem \ref{thm:K(1)coeff}. Therefore, the element $\mu_3$ must support a differential. The only possibility for bidegree reasons is
\[
d_{p^2}(\mu_3)\dot{=}v_1^{p^2}\lambda_2.
\]
Thus, we obtain 
\[
v_1^{-1}E_{p^2+1}\cong K(1)_*\otimes E(\lambda_1, \lambda_3,\lambda_4^{\prime})\otimes P(\mu_3^p)
\]
where 
\[
\lambda_4^{\prime}:= \lambda_2\mu_3^{p-1}.
\]
So the bidegree of $\lambda_4^{\prime}$ is given by 
\[
|\lambda_4^{\prime}|= (2p^4-2p^3+2p^2-1,0).
\]
For analogous reasons, the class $\lambda_4'$ is a permanent cycle, and $\mu_3^p$ cannot be a permanent cycle. Based on degree considerations, there are two possible differentials, 
\[
\begin{array}{ccc}
d_{p^2}(\mu_3^p)\dot{=} v_1^{p^2}\lambda_4' &\text{ or } &d_{p^3}(\mu_3^p)\dot{=}v_1^{p^3}\lambda_3.
\end{array}
\]
The first would contradict the Leibniz rule for $d_{p^2}$ and the fact that $d_{p^2}(\mu_3)\dot{=}v_1^{p^2}\lambda_2$. Thus, 
\[
v_1^{-1}E_{p^3+1}\cong K(1)_*\otimes E(\lambda_1, \lambda_4^{\prime}, \lambda_5^{\prime})\otimes P(\mu_3^{p^2})
\]
where
\[
\lambda_5^{\prime}:= \lambda_3\mu_3^{p(p-1)} .
\]
The bidegree of $\lambda_5'$ is 
\[
|\lambda_5'|=(2p^5-2p^4+2p^3-1,0).
\]
For degree reasons, the class $\lambda_5'$ is a permanent cycle. As before, the class $\mu_3^{p^2}$ must support a differential. Degree considerations, again, give two possibilities
\[
d_{p^3}(\mu_3^{p^2})\dot{=}v_1^{p^3}\lambda_5'
\]
or 
\[
d_{p^4+p^2}(\mu_3^{p^2}) \dot{=} v_1^{p^4+p^2}\lambda_4'.
\]
The former would contradict the Leibniz rule, leaving the latter as the only possibility. This gives us
\[
v_1^{-1}E_{p^4+p^2+1}\cong K(1)_*\otimes E(\lambda_1, \lambda_5', \lambda_6')\otimes P(\mu_3^{p^3})
\]
where $\lambda_6':= \lambda_4'\mu_3^{p^2(p-1)}$. We will continue via induction. First we need some notation. We will recursively define classes $\lambda_n^{\prime}$ by 
\[
\lambda_n^{\prime}:= \begin{cases}
	\lambda_n & n=1,2,3\\
	\lambda_{n-2}'\mu_3^{p^{n-4}(p-1)} & n\geq 4
\end{cases}
\]
We let $d'(n)$ denote the topological degree of $\lambda_n'$. Then this function is given recursively by 
\[
d'(n) = \begin{cases}
	2p^n-1 & n=1,2,3\\
	2p^n-2p^{n-1}+d(n-2) & n>3
\end{cases}
\]
Thus, by a simple induction, one has 
\[
d'(n) = \begin{cases}
	2p^n-1 & n=1,2,3\\
	2p^n-2p^{n-1}+2p^{n-2}-2p^{n-3}+\cdots + 2p^2-1 & n\equiv 0\mod 2,\, n>3\\
	2p^n-2p^{n-1}+ 2p^{n-2}-2p^{n-3}+\cdots + 2p^3-1& n\equiv 1\mod 2, \, n>3
\end{cases}.
\]
Observe that the integers $2p^{n+1}-d(n)-1$ and $2p^{n+1}-d(n+1)-1$ are both divisible by $|v_1|$. Let $r'(n)$ denote the integer 
\[
r'(n):=|v_1|^{-1}(|\mu_3^{p^{n-1}}-|\lambda_n'|-1)=|v_1|^{-1}(2p^{n+2}-d'(n+1)-1).
\] 
Then a simple induction shows that 
\[
r'(n) = \begin{cases}
	p^{n+1}+p^{n-1}+p^{n-3} +\cdots + p^2 & n\equiv 1 \mod 2\\
	p^{n+1}+p^{n-1}+p^{n-3}+\cdots + p^3 & n\equiv 0 \mod 2
\end{cases}.
\]



We can now describe the differentials in the $v_1$-BSS. 


\begin{thm}\label{key to proof 2}
	In the $v_1$-BSS, the following hold: 
	\begin{enumerate}
		\item The only nonzero differentials are in $v_1^{-1}E_{r'(n)}$. 
		\item The $r'(n)$th page is given by 
		\[
		v_1^{-1}E_{r'(n)} \cong K(1)_*\otimes E(\lambda_1, \lambda_n', \lambda_{n+1}')\otimes P(\mu_3^{p^{n-1}})
		\]
		and the classes $\lambda_{n}', \lambda_{n+1}'$ are permanent cycles. 
		\item The differential $d_{r'(n)}$ is uniquely determined by multiplicativity of the BSS and the differential
		\[
		d_{r'(n)}(\mu_3^{p^{n-1}})\dot{=}v_1^{r'(n)}\lambda_n'.
		\]
	\end{enumerate}
\end{thm}
\begin{proof}
	We proceed by induction. We have already shown the theorem for $n\leq 4$. Assume inductively that 
	\[
	v_1^{-1}E_{r'(n)}\cong K(1)_*\otimes E(\lambda_1, \lambda_n', \lambda_{n+1}')\otimes P(\mu_3^{p^{n-1}}).
	\]
	By the inductive hypothesis, $\lambda_n'$ is an infinite cycle.  
	
	Since $\lambda_n', \lambda_{n+1}'$ are both in odd topological degree, $\lambda_{n+1}'$ cannot support a differential hitting the $v_1$-towers on $\lambda_i^{\prime}$ for $i<n+1$ . Thus, the only possibility is that $\lambda_{n+1}'$ supports a differential into the $v_1$-tower on 1 or $\lambda_1$. But this would contradict Theorem \ref{thm:K(1)coeff}. Therefore, the class $\lambda_{n+1}'$ is a permanent cycle. 
	
	The class $\mu_3^{p^{n-1}}$ must support a differential, for if it did not, then the spectral sequence would collapse. This would lead to a contradiction of Theorem \ref{thm:K(1)coeff}. Degree considerations show that the following differentials are possible
	\[
	d_{k(n)}(\mu_3^{p^{n-1}})\dot{=}v_1^{k(n)}\lambda_{n+1}'
	\]
	and 
	\[
	d_{r(n)}(\mu_3^{p^{n-1}})\dot{=}v_1^{r'(n)}\lambda_{n}'
	\]
	where
	\[
	\ell(n) = |v_1^{-1}|(2p^{n+2}-|\lambda_{n+1}'|).
	\]
	An elementary inductive computation shows that 
	\[
	\ell(n) = r'(n-1).
	\]
	The former differential cannot occur, for by the inductive hypothesis, 
	\[
	d_{r'(n-1)} (\mu_3^{p^{n-2}})\dot{=}\lambda_{n-1}', 
	\]
	and the former differential would contradict the Leibniz rule. So we must conclude the latter differential occurs. This concludes proof.
\end{proof}

We now state the main result of this section. 
\begin{thm}\label{mod p v_2}
For each $n \ge 2$ and each nonnegative integer $m$ with $m \not\equiv p - 1 \mod{p}$ there are elements $z_{n,m}$ and $z^{\prime}_{n,m}$ in $THH_*(\B;k(1))$ such that
\begin{enumerate} 
\item $z_{n,m}$ projects to $\lambda^{\prime}_n\mu_3^{mp^{n-2}}$ in $E_{\infty}^{*,0}$
\item $z_{n,m}^{\prime}$ projects to $\lambda^{\prime}_n\lambda^{\prime}_{n+1}\mu_3^{mp^{n-2}}$ in $E_{\infty}^{*,0}$
\end{enumerate}
 As a $P(v_1)$-module, $THH_*(\B;k(1))$ is generated by the unit element
$1$, $\lambda_1$, and the elements $z_{n,m},z_{n,m}^{\prime}$. The only relations are
\[v_1^{r^{\prime}(n)}z_{n,m}=v_1^{r^{\prime}(n)}z_{n,m}^{\prime}=0.\]
\end{thm}
To prove this, we first need to prove a lemma. 
\begin{lem} \label{lem mod p v_2}
For $r\ge 2$ with $n=2$ and $k=1$, the $E_r(THH(\B))$-page of the Adams spectral sequence \eqref{qx} is generated by elements in filtration $0$ as a $P(1)$-module and $E_{r}^{*,*}$ is a direct sum of copies of $P(1)$ and $P(1)_i$ for $i\le r$. 
\end{lem}
\begin{proof}
We will begin by proving the first statement by induction. Note that \eqref{v_1bockinput} implies the base case in the induction when $r=2$. Suppose the statement holds for some $r$. Choose a basis $y_i$ for the $\mathbb{F}_p$-vector space $V_r$ such that $V_r=\{ x \in E_r^{*,0}\mid v_1^{r-1}x=0\}$. Then $d_r(y_i)$ is in filtration $r$ and since the differentials are $v_1$-linear, $v_1^{r-1}d_r(y_i)=0$. However, this contradicts the induction hypothesis because the induction hypothesis implies that all elements in filtration $r$ are $v_1$-torsion-free. Thus, each basis element $y_i$ is a $d_r$-cycle. Next choose a set of elements $\{y^{\prime}_j\}\subset E_r^{*,0}$ such that $\{d_r(y_j^{\prime})\}$ is a basis for $\im(d_r\co E_r^{*,0}\to E_r^{*,r})$. Choose $y^{\prime\prime}_j\in E_r^{*,0}$ such that $v_1^{r}y^{\prime\prime}_j=d_r(y_j^{\prime})$. Then $y^{\prime\prime}_j$ are $d_r$-cycles and $y^{\prime \prime}_j$ and $y_j$ are linearly independent. We can therefore choose $d_r$-cycles $y^{\prime \prime\prime}_j$ such that $\{y_j\}\cup\{y_j^{\prime\prime}\}\cup\{y_j^{\prime\prime\prime}\}$ are a basis for the $d_r$-cycles in $E_{r}^{*,0}$. Then 
$\{y_j\}\cup\{y_j^{\prime}\}\cup \{y_j^{\prime\prime}\}\cup\{y_j^{\prime\prime\prime}\}$
are a basis for $E_r^{*,0}$ and the differential is completely determined by the formulas
\[ \begin{array}{cccc} d_r(y_i)=0 , &d_r(y_j^{\prime})=v_1^{r}y_j^{\prime \prime}, & d_r(y_j^{\prime \prime})=0, \text{ and } & d_r(y_j^{\prime \prime \prime})=0. \end{array}\]
Thus, $E_r^{*,*}$ is generated as a $P(1)$-module by $y_i,$ $y_i^{\prime \prime}$, and $y_i^{\prime \prime\prime}$ where $v_1^{r-1}y_i=0$ and $v_1^ry_i^{\prime \prime}=0$ and $y_i^{\prime \prime \prime}$ is $v_1$-torsion free. 
\end{proof}
\begin{proof}[Proof of Theorem \ref{mod p v_2}]
For brevity, we will let $\delta_{n,m}$ denote $\lambda_n^{\prime}\mu^{mp^{n-2}}$ and we will let $\delta_{n,m}^{\prime}$ denote $\lambda_n^{\prime}\lambda_{n+1}^{\prime}\mu^{mp^{n-2}}$.
By Lemma \ref{lem mod p v_2} and Lemma \ref{mod p v_n} it suffices to prove that the elements $\delta_{n,m}$, and $\delta_{n,m}^{\prime}$ are infinite cycles that, together with $1$ and $\lambda_1$, form a basis for $E_{\infty}^{*,0}$ as an $\mathbb{F}_p$-vector space, and that each of $\delta_{n,m}$, $\delta_{n,m}^{\prime}$ are killed by $v_1^{r^{\prime}(n)}$. By induction on $n$, we will prove
\[ E_{r(n)}(THH(\B))\cong M_n\oplus E(\lambda_1,\lambda_n^{\prime},\lambda_{n+1}^{\prime})\otimes P(\mu_3^{p^{n-1}})\]
where $M_n$ is generated by $\{\delta_{k,m}, \delta_{k,m}^{\prime}\mid k<n\}$ modulo the relations 
\[v_2^{r(k)}\delta_{k,m}=v_2^{r(k)}\delta_{k,m}^{\prime}=0. \]
This statement holds for $n=1$ by \eqref{v_1bockinput}. Assume the statement holds for all integers less than or equal to some $N\ge1$. Lemma \ref{lem mod p v_2}, Lemma \ref{mod p v_n} and Theorem \ref{key to proof} imply that the only nontrivial differentials with source in $E_{r(N)}^{*,0}$ are the differentials
\[ d_{r(N)}(\mu^{(m+1)r^{\prime}(N)})=(m+1)v_1^{r^{\prime}(N)}\lambda_{N}^{\prime}\mu^{mp^{N-1}}\dot{=}\delta_{N,m},\]
and the differentials 
\[ d_{r^{\prime}(N)}(\lambda_{N+1}^{\prime}\mu^{(m+1)r^{\prime}(N)})=(m+1)v_1^{r^{\prime}(N)}\lambda_n^{\prime}\lambda_{N+1}^{\prime}\mu^{mp^{N-2}}\dot{=}\delta_{N,m}^{\prime}\]
where $m\not \equiv p-1 \mod{p}$. Combining this with Lemma \ref{lem mod p v_2} and Lemma \ref{mod p v_n}, this implies that
\[ E_{r(N)+1}(\THH(\B))\cong M_n\oplus V_{N+1}\oplus \left ( P(2)\otimes E(\lambda_1, \lambda_{N}^{\prime},\lambda_{N}^{\prime}\mu_3^{(p-1)p^{N-2}} )\otimes P(\mu^{p^N})\right )\]
where $V_{N+1}$ has generators $\delta_{N,m}$ and $\delta_{N,m}^{\prime}$ and relations 
\[ v_2^{r(N)}\delta_{N,m}=v_2^{r(N)}\delta_{N,m}^{\prime}=0.\]
By Lemma \ref{lem mod p v_2}, Lemma \ref{mod p v_n} and Theorem \ref{key to proof 2} there is an isomorphism
\[ E_{r(N)+1}(\THH(\B))\cong E_{r(N+1)}(\THH(\B)).\] 
Also, note that $M_N\oplus V_{N+1}=M_{N+1}$ and $\lambda_{N}^{\prime}\mu_3^{(p-1)p^{N-2}}=\lambda_{N+2}^{\prime}$ by definition. This completes the inductive step and consequently the proof.
\end{proof}














 