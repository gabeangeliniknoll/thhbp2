% root file is THHBP2.tex

\section{The $v_1$-Bockstein spectral sequence}

In this section we will begin our analysis of the $v_1$-Bockstein spectral sequence for computing the homotopy of $\THH(R;k(1))$, i.e. the spectral sequence \eqref{v_1BSS}. Our approach is very similar to the one outlined in the previous section.

To start, we need to compute $K(1)_*\tBP{2}$. This requires determining $\eta_R(v_{2+n})$ in $K(1)_*BP$ modulo the ideal generated by $(\eta_R(v_3), \ldots, \eta_R(v_{1+n}))$. We will need the following. 

\begin{lem}\label{lem:rightunit}{\cite[Lemma A.2.2.5]{greenbook}} Let $v_n$ denote the Araki generators. Then there is the following equality in $BP_*BP$
\[
\sum_{i,j\geq 0}\hspace{-5pt}\mbox{}^F\: t_i\eta_R(v_j)^{p^i}=\sum_{i,j\geq 0}\hspace{-5pt}\mbox{}^F\:v_it_j^{p^i}
\]
\end{lem}

In our context, the distinction between Hazewinkel generators and Araki generators is unimportant, as the two sets of generators coincide modulo $p$. In $K(1)_*BP$, we have killed all $v_i$'s except $v_1$, which gives us the following equation
\[
\sum_{i,j\geq 0}\hspace{-5pt}\mbox{}^F\: t_i\eta_R(v_j)^{p^i}=\sum_{k\geq 0}\hspace{-4pt}\mbox{}^F\:v_1t_k^{p}
\]
Note that the following degrees of the terms:
\begin{align*}
	|v_1t_j^p|&= 2(p^{j+1}-1)\\
	|t_i\eta_R(v_j)^{p^i}|&= 2(p^{i+j}-1)
\end{align*}
Since we are interested in the term $\eta_R(v_{2+n})$, we collect all the terms on the left of degree $2(p^{2+n}-1)$. Thus we are summing over the ordered pairs $(i,j)$ such that $i+j=2+n$. Since we only care about $\eta_R(v_{2+n})$ modulo $\eta_R(v_3), \ldots, \eta_R(v_{1+n})$ we only need to collect the terms where $j=1, 2$, or $2+n$. This shows that 
\[
t_{1+n}\eta_R(v_1)^{p^{n+1}}+t_n\eta_R(v_2)^{p^n}+\eta_R(v_{n+2})=v_1t_{n+1}^p
\]
The value of $\eta_R$ on $v_1$ and $v_2$ can also be computed by Lemma \ref{lem:rightunit}. One obtains, in $K(1)_*BP$, the following
\begin{align*}
\eta_R(v_1)&=v_1\\
\eta_R(v_2)&=v_1t_1^p-t_1v_1^p.
\end{align*}
Combining these observations, we obtain

\begin{lem}
	In $K(1)_*BP$, the following congruence is satisfied
	\[
	\eta_R(v_{2+n})\equiv v_1t_{n+1}^p-v_1^{p^n}t_1^{p^{n+1}}t_n+v_1^{p^{n+1}}(t_1^{p^n}t_n-t_{n+1})  \mod(\eta_R(v_3), \ldots, \eta_R(v_{1+n}))
	\]
	for $n\geq 1$.
\end{lem}
\dom{URG!! that is not too pretty. }
\gabe{Perhaps leaving it as 
\[ \eta_R(v_{1+k})=v_1t_k^p-t_kv_1^{p^k}-t_{k-1}\eta_R(v_2)^{p^{k-1}} \mod(\eta_R(v_3),\ldots, \eta_R(v_{1+k-1}) ) \]
for $k\ge 2$ is simpler? }


Consequently, we have 

\begin{cor}
	There is an isomorphism of $K(1)_*$-algebras
	\[
	K(1)_*R\cong K(1)_*BP/(v_1t_{n+1}^p-v_1^{p^n}t_1^{p^{n+1}}t_n+v_1^{p^{n+1}}(t_1^{p^n}t_n-t_{n+1}) \mid n\geq 1)
	\]
\end{cor}

Let $u_n:=v_1^{\frac{p^n-1}{p-1}}t_n$. These elements are in degree 0, giving us an isomorphism
\[
K(1)_*R\cong_{K(1)_*}K(1)_*\otimes_{\F_p}K(1)_0R.
\]
The calculations above tell us 

\begin{cor}
	There is an isomorphism of $\F_p$-algebras
	\[
	K(1)_0 R\cong\F_p[t_i\mid i\geq 1]/(u_{n+1}^p-u_1^{p^{n+1}}u_n+u_1^{p^n}u_n-u_{n+1}\mid n\geq 1).
	\]
\end{cor}

Our goal is to use this and the $K(1)$-based B\"okstedt spectral sequence to compute the $K(1)$-homology of $\THH(R)$. This is a spectral sequence of the form 
\[
E^2_{s,t}= \HH^{K(1)_*}_s(K(1)_*R)\implies K(1)_{s+t}\THH(R).
\]
The above considerations tell us that the $E^2$-page is 
\[
E^2\cong K(1)_*\otimes \HH^{\F_p}(K(1)_0R). 
\]

The following will be useful for our calculation.

\begin{lem}[\cite{MilneLEC}]\label{lem:etale}
	Let $V = \Spec(A)$ be a nonsingular affine variety over a field $k$. Let $W$ be the subvariety of $V\times \mathbb{A}^n$ defined by equations
	\[
	g_i(Y_1, \ldots, Y_n)=0, \, g_i\in A[Y_1, \ldots, Y_n],\, i=1,\ldots , n.
	\]
	Then the projection map $W\to V$ is \'etale at a point $(P;b_1, \ldots, b_n)$ of $W$ if and only if the Jacobian matrix $\begin{pmatrix}
		\frac{\partial g_i}{\partial Y_j}
	\end{pmatrix}$ is a nonsingular matrix at $(P; b_1, \ldots , b_n)$.
\end{lem} 

\begin{thm}[\'Etale Descent, \cite{WeibelGeller}]
	Let $A\hookrightarrow B$ be an \'etale extension of commutative $k$-algebras. Then there is an isomorphism
	\[
	\HH_*(B)\cong \HH_*(A)\otimes_A B
	\]
\end{thm}





\dom{here is an idea for how to use this isomorphism. Hochschild homology behaves nicely for etale extensions, can we show that the finitely generated subalgebras of the above are etale over $\F_p$. I think this may be a good avenue since this is the essential point in the calculation of McClure and Staffeldt.}

\begin{ex}
	Consider the subalgebra
	\[
	\F_p[u_1,u_2]/(u_2^p-u_1^{p^2+1}+u_1^{p+1}-u_2=f_1).
	\]
	We will regard this as a $\F_p[u_1]$-algebra. The derivative $\partial_{u_2}f_1$ is $-1$, and therefore a unit at every point. Then Lemma \ref{lem:etale} tells us that this algebra is then \'etale over $\F_p[u_1]$. \dom{but we really want \'etale over $\F_p$. I have a feeling its not etale over $\F_p$.}
\end{ex} 
\gabe{I like this idea! Let me run with it a bit}


By the same argument given above, there are a sequence of sub-algebras $A_n$ of 
	\[
	K(1)_0 R\cong\F_p[t_i\mid i\geq 1]/(u_{n+1}^p-u_1^{p^{n+1}}u_n+u_1^{p^n}u_n-u_{n+1}\mid n\geq 1)=:A.
	\]
such that each map $A_i\rightarrow A_{i+1}$ is an \'etale map. By the \'etale base change formula for Hochschild homology, 
\[ \HH_*^{\F_p}(A_{i+1})\cong \HH_*^{\F_p}(A_i)\otimes_{A_i}A_{i+1}\]
and since $HH_*$ commutes with colimits of $k$-algebras, we have the following: 
	\[ 
	\begin{array}{rcl} 
		\HH_*^{\F_p}(A) & \cong &\HH_*^{\F_p}(\colim A_n) \\
				         & \cong & \colim  \HH_*^{\F_p}(A_n) \\
				         & \cong & \colim \HH_*^{\F_p}(A_1)\otimes_{A_1}A_n \\
				         & \cong & \HH_*^{\F_p}(A_1)\otimes_{A_1}A \\
	\end{array}
	\]
where we have the second to last isomorphism by an easy induction and the last isomorphism because colimits of $k$-algebras commute with $\otimes$. 
This shows that 
\[ \HH_*(K(1)_*R)\cong K(1)_*\otimes E(\sigma t_1)\otimes K_0(R) \]
and therefore, since $\sigma t_1=\lambda_1$, 
\[ K(1)_*\THH(R)\cong K(1)_*R\otimes E(\lambda_1) \]
and 
\[ \THH_*(R;K(1))\cong K(1)_*\otimes E(\lambda_1) \]
or in other words $\lambda_1$ is $P(v_1)$-free and the differentials will just go between the other two generators. 
\gabe{I still have to check a lot of parts of this argument, but it seems promising.}
	
We have the input needed to compute the following Eilenberg-Moore spectral sequence,
\[
\Tor^{K(1)_*R}(K(1)_*K(1), K(1)_*\THH(R))\implies K(1)_*\THH(R;K(1))
\]
From the previous computation, the $E^2$-term is concentrated in $\Tor_0$ and is 
\[
K(1)_*K(1)\otimes E(\lambda_1).
\]		         
Thus, every class besides $1$ and $\lambda_1$ is $v_1$-torsion in $\THH_*(R;k(1))$. Since $\THH(R;K(1))$ is a $K(1)$-module, this implies that 
\[
\THH(R;K(1))\simeq K(1)\vee \Sigma^{2p-1}K(1).
\]
In summary, we have shown
\begin{thm}\label{thm:K(1)coeff}\mbox{}
	\begin{enumerate}
		\item The $K(1)$-homology of $\THH(R;K(1))$ is $K(1)_*K(1)\otimes E(\lambda_1)$. 
		\item The only $v_1$-torsion free classes in $\THH(R;k(1))$ are $1$ and $\lambda_1$.
	\end{enumerate}
\end{thm}

\subsection{Differentials in the $v_1$-BSS}

We now analyze the $v_1$-BSS \eqref{v_1BSS}. Recall that this spectral sequence is of the form 
\[
\THH(R;\F_p)[v_1]\implies \THH(R;k(1)).
\]
Thus the $E_1$-page is 
\[
K(1)_*\otimes E(\lambda_1, \lambda_2, \lambda_3)\otimes P(\mu_3).
\]
Since the $\lambda_i$ are all in odd total degree and since $1$ is to be $v_1$-torsion free, the $\lambda_i$ are all permanent cycles. If $\mu_3$ were a permanent cycle, then by multiplicativity, the spectral sequence would collapse, and this would contradict Theorem \ref{thm:K(1)coeff}. So we must have that $\mu_3$ supports a differential. The only possibility is 
\[
d_{p^2}(\mu_3)\dot{=}v_1^{p^2}\lambda_2.
\]
Thus, one obtains
\[
v_1^{-1}E_{p^2+1}\cong K(1)_*\otimes E(\lambda_1, \lambda_3,\lambda_4)\otimes P(\mu_3^p)
\]
where 
\[
\lambda_4':= \lambda_2\mu_3^{p-1}.
\]
So the bidegree of $\lambda_4'$ is given by 
\[
|\lambda_4'|= (2p^4-2p^3+2p^2-1,0).
\]
For analogous reasons, the class $\lambda_4'$ is a permanent cycle, and $\mu_3^p$ cannot be a permanent cycle. Based on degree considerations, there are two possible differentials, 
\[
d_{p^2}(\mu_3^p)\dot{=} v_1^{p^2}\lambda_4'
\]
or 
\[
d_{p^3}(\mu_3^p)\dot{=}v_1^{p^3}\lambda_3.
\]
The first would contradict the Leibniz rule for $d_{p^2}$ and the fact that $d_{p^2}(\mu_3)\dot{=}v_1^{p^2}\lambda_2$. This leaves the second as the only possibility. Thus 
\[
v_1^{-1}E_{p^3+1}\cong K(1)_*\otimes E(\lambda_1, \lambda_4', \lambda_5')\otimes P(\mu_3^{p^2})
\]
where
\[
\lambda_5:= \lambda_3\mu_3^{p(p-1)} .
\]
The bidegree of $\lambda_5'$ is 
\[
|\lambda_5'|=(2p^5-2p^4+2p^3-1,0).
\]
For degree reasons, the class $\lambda_5'$ is a permanent cycle. As before, the class $\mu_3^{p^2}$ must support a differential. Degree considerations, again, give two possibilities
\[
d_{p^3}(\mu_3^{p^2})\dot{=}v_1^{p^3}\lambda_5'
\]
or 
\[
d_{p^4+p^2}(\mu_3^{p^2}) \dot{=} v_1^{p^4+p^2}\lambda_4'.
\]
The former would contradict the Leibniz rule, leaving the latter as the only possibility. This gives us
\[
v_1^{-1}E_{p^4+p^2+1}\cong K(1)_*\otimes E(\lambda_1, \lambda_5', \lambda_6')\otimes P(\mu_3^{p^3})
\]
where $\lambda_6':= \lambda_4'\mu_3^{p^2(p-1)}$. We will continue via induction. First we need some notation. We will recursively define classes $\lambda_n'$ by 
\[
\lambda_n':= \begin{cases}
	\lambda_n & n=1,2,3\\
	\lambda_{n-2}'\mu_3^{p^{n-4}(p-1)} & n\geq 4
\end{cases}
\]
We let $d'(n)$ denote the topological degree of $\lambda_n'$. Then this function is given recursively by 
\[
d'(n) = \begin{cases}
	2p^n-1 & n=1,2,3\\
	2p^n-2p^{n-1}+d(n-2) & n>3
\end{cases}
\]
Thus, by a simple induction, one has 
\[
d'(n) = \begin{cases}
	2p^n-1 & n=1,2,3\\
	2p^n-2p^{n-1}+2p^{n-2}-2p^{n-3}+\cdots + 2p^2-1 & n\equiv 0\mod 2,\, n>3\\
	2p^n-2p^{n-1}+ 2p^{n-2}-2p^{n-3}+\cdots + 2p^3-1& n\equiv 1\mod 2, \, n>3
\end{cases}.
\]
Observe that the integers $2p^{n+1}-d(n)-1$ and $2p^{n+1}-d(n+1)-1$ are both divisible by $|v_1|$. Let $r'(n)$ denote the integer 
\[
r'(n):=|v_1|^{-1}(|\mu_3^{p^{n-1}}-|\lambda_n'|-1)=|v_1|^{-1}(2p^{n+2}-d'(n+1)-1).
\] 
Then a simple induction shows that 
\[
r'(n) = \begin{cases}
	p^{n+1}+p^{n-1}+p^{n-3} +\cdots + p^2 & n\equiv 1 \mod 2\\
	p^{n+1}+p^{n-1}+p^{n-3}+\cdots + p^3 & n\equiv 0 \mod 2
\end{cases}.
\]



We can now describe the differentials in the $v_1$-BSS. 


\begin{thm}
	In the $v_1$-BSS, one has 
	\begin{enumerate}
		\item The only nonzero differentials are in $v_1^{-1}E_{r'(n)}$. 
		\item The $r'(n)$th page is given by 
		\[
		v_1^{-1}E_{r'(n)} \cong K(1)_*\otimes E(\lambda_1, \lambda_n', \lambda_{n+1}')\otimes P(\mu_3^{p^{n-1}})
		\]
		and the classes $\lambda_{n}', \lambda_{n+1}'$ are permanent cycles. 
		\item The differential $d_{r'(n)}$ is uniquely determined by multiplicativity of the BSS and the differential
		\[
		d_{r'(n)}(\mu_3^{p^{n-1}})\dot{=}v_1^{r'(n)}\lambda_n'.
		\]
	\end{enumerate}
\end{thm}
\begin{proof}
	We proceed by induction. We have already shown the theorem for $n\leq 4$. Assume inductively that 
	\[
	v_1^{-1}E_{r'(n)}\cong K(1)_*\otimes E(\lambda_1, \lambda_n', \lambda_{n+1}')\otimes P(\mu_3^{p^{n-1}}).
	\]
	By inductive hypothesis, $\lambda_n'$ is a permanent cycle.  
	\gabe{Again, you are saying $\lambda_n'$ is a permanent cycle in the $v_1$-inverted spectral sequence, but this can't be the case. I think you mean that it is a permanent cycle before inverting $v_1$.}
	
	Since $\lambda_n', \lambda_{n+1}'$ are both in odd topological degree, $\lambda_{n+1}'$ cannot support a differential into the $v_1$-towers on $\lambda_n'$. Thus the only possibility is that $\lambda_{n+1}'$ supports a differential into the $v_1$-tower on 1 or $\lambda_1$. But this would contradict Theorem \ref{thm:K(1)coeff}. So $\lambda_{n+1}'$ is a permanent cycle. 
	
	The class $\mu_3^{p^{n-1}}$ must support a differential, for if it did not, then the spectral sequence would collapse. This would lead to a contradiction of Theorem \ref{thm:K(1)coeff}. Degree considerations show that the following differentials are possible
	\[
	d_{k(n)}(\mu_3^{p^{n-1}})\dot{=}v_1^{k(n)}\lambda_{n+1}'
	\]
	and 
	\[
	d_{r(n)}(\mu_3^{p^{n-1}})\dot{=}v_1^{r'(n)}\lambda_{n}'
	\]
	where
	\[
	k(n) = |v_1^{-1}|(2p^{n+2}-|\lambda_{n+1}'|).
	\]
	An elementary inductive computation shows that 
	\[
	k(n) = r'(n-1).
	\]
	The former differential cannot occur, for by the inductive hypothesis, 
	\[
	d_{r'(n-1)} (\mu_3^{p^{n-2}})\dot{=}\lambda_{n-1}', 
	\]
	and the former differential would contradict the Leibniz rule. So we must conclude the latter differential occurs. This concludes proof.
\end{proof}

\todo{Prove the analogue of Theorem 7.1 of \cite{McClureStaffeldt}}


%
%To compute $\THH(R;k(1))$, we will compare it to $\THH(\ell;k(1))$. Since there is a map of commutative ring spectra
%\[
%R\to \ell
%\]
%obtained from coning off $v_2$, we have an induced map in topological Hochschild homology with $\F_p$ coefficients,
%\[
%\varphi:\THH(R;\F_p)\to \THH(\ell;\F_p).
%\]
%We determine this map. From the B\"okstedt spectral sequence or \cite[Corollary 3.2]{AHL} we computed the homotopy groups of $\THH(R;\F_p)$, \eqref{eqn:THH(R;F_p)}. Similarly, one finds that 
%\[
%\THH_*(\ell;\F_p)\cong E(\lambda_1,\lambda_2)\otimes P(\widetilde{\mu})
%\] 
%In the K\"unneth spectral sequences computing these homotopy groups, the elements $\lambda_i$ are represented by $\sigma\zeta_i$ when $p$ is odd or $\sigma\zeta_i^2$ when $p=2$. The class $\mu$ is represented by $\sigma\otau_3$ or $\sigma\zeta_4$ and $\widetilde{\mu}$ is represented by $\sigma\otau_2$ or $\sigma\zeta_3$. Furthermore, there are the following hidden extensions in the spectral sequence 
%\[
%(\gamma_i^0)^p=\gamma_{i+1}^0.
%\]
%
%Combining these observations furnishes the following proposition.
%
%\begin{prop}
%	The map $\varphi$ induces the following in map in homotopy,
%	\[
%	\varphi_*:\THH_*(R;\F_p)\to \THH_*(\ell;\F_p)
%	\]
%	\begin{align*}
%		\lambda_1&\mapsto \lambda_1\\
%		\lambda_2&\mapsto \lambda_2\\
%		\lambda_3&\mapsto 0\\
%		\mu      &\mapsto \widetilde{\mu}^p
%	\end{align*}
%\end{prop}
%
%This allows us to deduce many differentials in the $v_1$-BSS for $R$. We need to recall that the differentials in the $v_1$-BSS of $\ell$. We must first introduce some notation. Recursively define a function $r(n)$ by $r(1)=p$ and $r(2)=p^2$ and $r(n) = p^n+r(n-2)$ for $n\geq 3$. Define $\kappa_i:= \lambda_{i-2}$...
%
%\begin{thm}[\cite{AHL}]
%	\end{thm}
%
%\dom{Actually, trying to establish differentials in the $v_1$-BSS for $\THH(R;k(1))$ will require us knowing first what classes are $v_1$-torsion. }














 