\documentclass[12pt]{amsart}
\usepackage[margin=1in]{geometry}
\usepackage{amsmath}
\usepackage{amsthm}
\usepackage{amssymb}
\usepackage{lscape,xcolor}
\usepackage{graphicx}
\usepackage{mathrsfs}
%\usepackage{mathrools}
\usepackage{stmaryrd}
\usepackage{verbatim}
\usepackage{rotating}
\usepackage{tikz-cd}
\usepackage{xypic}
\usepackage{amsrefs}
\usepackage{hyperref}
\usepackage{euscript}
\usepackage[colorinlistoftodos]{todonotes}
\usepackage{spectralsequences}


\usepackage[sc]{mathpazo}
\linespread{1.05}         % Palatino needs more leading (space between lines)
\usepackage[T1]{fontenc}

\usepackage[OT2,T1]{fontenc}
\newcommand\textcyr[1]{{\fontencoding{OT2}\fontfamily{wncyr}\selectfont #1}}


%\usepackage{luasseq}
\usepackage{xcolor}
\definecolor{seagreen}{RGB}{46,139,87}
\definecolor{maroon}{RGB}{128,0,0}
\definecolor{darkviolet}{RGB}{148,0,211}
\definecolor{twelve}{RGB}{100,100,170}
\definecolor{thirteen}{RGB}{100,150,50}
\definecolor{fourteen}{RGB}{200,0,0}
\definecolor{fifteen}{RGB}{0,200,0}
\definecolor{sixteen}{RGB}{0,0,200}
\definecolor{seventeen}{RGB}{200,0,200}
\definecolor{eighteen}{RGB}{0,200,200}



\parskip 0.7pc
\parindent 0pt

\allowdisplaybreaks[1]

%%%%%%%%%%%%%%% Basic commands %%%%%%%%%%%%%%%%%%
\newcommand{\dotequiv}{\overset{\scriptstyle{\centerdot}}{\equiv}}
\newcommand{\nd}{\not\!|}
\newcommand{\mmod}{\! \sslash \!}

\newcommand{\mc}[1]{\mathcal{#1}}
\newcommand{\ull}[1]{\underline{#1}}
\newcommand{\mb}[1]{\mathbb{#1}}
\newcommand{\mr}[1]{\mathrm{#1}}
\newcommand{\mbf}[1]{\mathbf{#1}}
\newcommand{\mit}[1]{\mathit{#1}}
\newcommand{\mf}[1]{\mathfrak{#1}}
\newcommand{\ms}[1]{\mathscr{#1}}
\newcommand{\abs}[1]{\lvert #1 \rvert}
\newcommand{\norm}[1]{\lVert #1 \rVert}
\newcommand{\bra}[1]{\langle #1 \rangle}
\newcommand{\br}[1]{\overline{#1}}
\newcommand{\brr}[1]{\overline{\overline{#1}}}
\newcommand{\td}[1]{\widetilde{#1}}
\newcommand{\tdd}[1]{\widetilde{\widetilde{#1}}}
\newcommand{\Z}{\mathbb{Z}}
\newcommand{\R}{\mathbb{R}}
\newcommand{\C}{\mathbb{C}}
\newcommand{\Q}{\mathbb{Q}}
\newcommand{\W}{\mathbb{W}}
\newcommand{\F}{\mathbb{F}}
\newcommand{\G}{\mathbb{G}}
\newcommand{\MS}{\mathbb{S}}
\newcommand{\PP}{\mathbb{P}}

\newcommand{\euscr}[1]{\EuScript{#1}}

%%%%%%%%%%%%%%%%% Spectra %%%%%%%%%%%%%%%

\newcommand{\tBP}[1]{BP\bra{#1}}
\newcommand{\AF}{\mr{AF}}
\newcommand{\TAF}{\mathrm{TAF}}
\newcommand{\TMF}{\mathrm{TMF}}
\newcommand{\Tmf}{\mathrm{Tmf}}
\newcommand{\tmf}{\mathrm{tmf}}
\newcommand{\bo}{\mathrm{bo}}
\newcommand{\bsp}{\mathrm{bsp}}
\newcommand{\HZ}{\mr{H}\Z}
\def \HF2{\mr{H}\F_2}
\newcommand{\bu}{\mr{bu}}
\newcommand{\MU}{\mr{MU}}
\newcommand{\KU}{\mr{KU}}
\newcommand{\KO}{\mr{KO}}
\newcommand{\EO}{\mr{EO}}
\newcommand{\BP}{\mr{BP}}
\newcommand{\K}{\mr{K}}
\newcommand{\ku}{\mathrm{ku}}

%%%%%%%%%%%%%%% Operators %%%%%%%%%%%%%%

\DeclareMathOperator{\Ext}{Ext}
\DeclareMathOperator{\Tor}{Tor}
\DeclareMathOperator{\aut}{Aut}
\DeclareMathOperator{\im}{im}
\DeclareMathOperator{\Sta}{Sta}
\DeclareMathOperator{\Map}{Map}
\DeclareMathOperator*{\holim}{holim}
\DeclareMathOperator*{\hocolim}{hocolim}
\DeclareMathOperator*{\colim}{colim}
\DeclareMathOperator*{\Tot}{Tot}
\DeclareMathOperator{\Spf}{Spf}
\DeclareMathOperator{\Aut}{Aut}
\DeclareMathOperator{\Spec}{Spec}
\DeclareMathOperator{\Proj}{Proj}

\DeclareMathOperator{\THH}{THH}

\DeclareMathOperator{\sq}{Sq}
\newcommand{\xib}{{\bar{\xi}}}
\newcommand{\s}{\wedge}
\newcommand{\Si}{\Sigma}
\newcommand\floor[1]{\lfloor#1\rfloor}

%%%%%%%%%%%%% Steenrod Algebra & Brown-Gitler Modules %%%%%%%%%%%

\newcommand{\A}{\ms{A}}
\newcommand{\sE}{\ms{E}}
\newcommand{\HZu}{\ull{\HZ}}
\newcommand{\bou}{\ull{\bo}}
\newcommand{\tmfu}{\ull{\tmf}}
\newcommand{\tBPu}[1]{\ull{\tBP{#1}}}
\newcommand{\buu}{\ull{\bu}}
\def \AA0{\br{A \mmod A(0)}_*}
\def \AA2{A\mmod A(2)_*}
\def \AE2{A\mmod E(2)_*}
\renewcommand{\AE}[1]{A\mmod E(#1)_*}
\DeclareMathOperator{\wt}{\mathrm{wt}}
\def \E2E1{(E(2)\mmod E(1))_*}



%%%%%%%%%%%%%%%% Categories %%%%%%%%%%%%%

\newcommand{\Top}{\mathsf{Top}}
\newcommand{\Operad}{\mathsf{Operad}}
\newcommand{\Alg}{\mathsf{Alg}}
\newcommand{\Monad}{\mathsf{Monad}}
\newcommand{\Set}{\mathsf{Set}}
\newcommand{\sSet}{\mathsf{sSet}}
\newcommand{\Man}{\mathsf{Man}}
\newcommand{\Presheaf}{\mathsf{Presheaf}}
\newcommand{\Fun}{\mathsf{Fun}}
\newcommand{\Grpd}{\mathsf{Grpd}}
\newcommand{\op}{\mathrm{op}}


%%%%%%%%%%%%%%% Homological Algebra %%%%%%%

\newcommand{\cone}[1]{\mathrm{cone}\left(#1\right)}


%%%%%%% for numbered theorems %%%%%%%%%
 \newtheorem{thm}[equation]{Theorem}
 \newtheorem{cor}[equation]{Corollary}
 \newtheorem{lem}[equation]{Lemma}
 \newtheorem{prop}[equation]{Proposition}
 \newtheorem{obs}[equation]{Observation}
  \newtheorem{rem}[equation]{Remark}
 
 \newtheorem*{thm*}{Theorem}
 \newtheorem*{cor*}{Corollary}
 \newtheorem*{lem*}{Lemma}
 \newtheorem*{prop*}{Proposition}
  \newtheorem*{not*}{Notation}

 
 \theoremstyle{definition}
 \newtheorem{defn}[equation]{Definition}
 \newtheorem{ex}[equation]{Example}
 \newtheorem{exs}[equation]{Examples}
 \newtheorem{rmk}[equation]{Remark}
\newtheorem{claim}[equation]{Claim}
 \newtheorem{question}[equation]{Question}
 \newtheorem{conjecture}[equation]{Conjecture}
%%%%%%%%%%%%%%%%%%%%%%%%%%%%%%%%%%%%%%%%

\newtheorem*{defn*}{Definition}
\newtheorem*{ex*}{Example}
\newtheorem*{exs*}{Examples}
\newtheorem*{rmk*}{Remark}
\newtheorem*{claim*}{Claim}
\newtheorem*{conventions}{Conventions}
\numberwithin{equation}{section}
\numberwithin{figure}{section}



\title{THH of $\tBP{2}$ with coefficients in $k(2)$}
\author{ G.~ Angelini-Knoll \"und D.~ Culver}\address{University of Illinois, Urbana-Champaign}\email{dculver@nd.edu}\address{Michigan State University, East Lansing}\email{angelini@math.msu.edu}

\begin{document}

\maketitle

%%%%%%%%%%%%%%%%%%
%%Copy and paste into current draft 
%%%%%%%%%%%%%%%%%%%
Let $\tBP{2}$ be the truncated Brown-Peterson spectrum with coefficients $\tBP{2}_*\cong \Z_{(p)}[v_1,v_2]$ and let $k(2)$ be connective Morava K-theory with coefficients $k(2)_*\cong \F_p[v_2^{\pm1}]$. At the moment, we will let $p$ be any prime number. Whenever we assume that there is a model for $\tBP{2}$ that is $E_{\infty}$ we will assume that $p=2$ or $3$. 

The goal is to compute $THH(\tBP{2};k(2))$ via the Bockstein spectral sequence 
\[ THH_*(\tBP{2};H\F_p)[v_2]\Rightarrow THH(\tBP{2};k(2)).\]

The first goal is to show
\begin{equation}\label{K(2) coeff} K(2)_*\cong THH_*(\tBP{2};K(2)) \end{equation}
which will imply that all the classes except the classes in the subalgebra $P(v_2)\subset THH_*(\tBP{2};H\F_p)[v_2]$ are $v_2$-torsion and this will force differentials in the spectral sequence. Our approach is entirely analogous to the calculation of McClure-Staffeldt except for one minor difference, which we will point out. 

To compute $THH_*(\tBP{2};K(2))$, we can first compute 
\[ K(2)_*THH(\tBP{2};K(2)) \]
and then use the fact that $THH(\tBP{2};K(2))$ is a free $K(2)$-module (since $K(2)$ is a field spectrum) and the collapse of the $K(2)$-based Adams spectral sequence to finish the computation. 

We use the $K(2)$-based B\"okstedt spectral sequence to compute $K(2)_*THH(\tBP{2};K(2))$; i.e. the spectral sequence
\[ HH_*^{K(2)_*}(K(2)_*\tBP{2};K(2))\Rightarrow K(2)_*THH(\tBP{2};K(2)).\]
the first goal will be to compute the input. 
\begin{lem}There is an isomorphism of graded rings
\[K(n)_*\tBP{n}\cong K(n)_*[t_1,t_2, \dots]/(v_nt_k^{p^n}-v_n^{p^k}t_k | k\ge 1).\]
\end{lem}
\begin{proof}
We adapt the proof in McClure-Staffeldt. First $K(n)_*BP\cong K(n)_*\otimes_{BP_*}BP_*BP$ because $BP$ is Landweber exact. Furthermore, $K(n)_*\otimes_{BP_*}BP_*BP\cong K(n)_*[t_1,t_2,\dots]$ and we can restrict $\eta_R:BP_*\rightarrow BP_*BP$ to $K(n)_*\otimes_{BP_*}BP_*BP$ to produce the map $\bar{\eta}_R$ and by Ravenel 
\begin{equation}\label{Rav form} \bar{\eta}_R(v_{n+k})=v_nt_k^{p^n}-v_n^{p^k}t_k  \text{ mod } (\bar{\eta}_R(v_{n+1}),\bar{\eta}_R(v_{n+2}), \dots\bar{\eta}_R(\dots, v_{n+k-1}) )\end{equation}
We can then construct $\tBP{n}$ using Baas-Sullivan theory and the effect is that 
\[ K(n)_*\tBP{n}\cong K(n)_*\otimes_{BP_*}BP_*BP/(\bar{\eta}_R(v_{n+1}), \bar{\eta}_R(v_{n+2}), \dots )\]
or in other words, by \eqref{Rav form}
\[ K(n)_*\tBP{n} \cong K(n)_*[t_1,t_2, \dots]/(v_nt_k^{p^n}-v_n^{p^k}t_k | k\ge 1) \]
as desired. 
\end{proof}
\begin{rem}
Note that the model of $\tBP{2}$ that we used in this lemma is not the same model that gives you an $E_{\infty}$-structure, but since there is a weak equivalence $\tBP{2}\simeq tmf_1(3)$ the map $K(n)_*\tBP{2}\cong K(n)_*tmf_1(3)$ as $K(n)_*$-modules. (\textcolor{seagreen}{Is the equivalence $\tBP{2}\simeq tmf_1(3)$ known to be an equivalence of $E_2$-algebras? Does this reasoning make sense to you?})
\end{rem}
We now describe the structure of $K(2)_*[t_1,t_2, \dots]/(v_2t_k^{p^2}-v_2^{p^k}t_k | k\ge 1)$. Note that it can be written as 
\begin{equation}\label{K(2) computation} K(2)_*[t_1,t_2, \dots]/(v_2t_k^{p^2}-v_2^{p^k}t_k | k\ge 1) \cong \bigotimes_{k\ge 1} K(2)_*[t_k]/(v_2t_k^{p^2}-v_2^{p^k}t_k )\end{equation}
where the tensor is taken over $K(2)_*$. Note that $|t_k|=2p^k-2$ and $2p^2-2|2p^k-2$ when $2|k$ and $2p-2|2p^k-2$ for all $k$. Therefore, 
\[ K(2)_*[t_k]/(v_2t_k^{p^2}-v_2^{p^k}t_k )\cong K(2)_*\otimes \mathbb{F}_p[u_k]/(u_k^{p^2}-v_2^{p^k-1}u_k]\]
where $u_k=t_kv_2^{m(k)}$ where $m(k)=-p^{k-2}-p^{k-4} - \dots p^2-1$ when $2|k$ and 
\[ K(2)_*[t_k]/(v_2t_k^{p^2}-v_2^{p^k}t_k )\cong K(2)_*\otimes \mathbb{F}_p[w_k]/(w_k^{p^2}-v_2^{p^k-1}w_k]\]
where $w_k=t_kv_2^{\ell(k)}$ where $\ell(k)=-p^{k-2}-p^{k-4} -\dots -p$ and $k$ is odd so that $|w_k|=2p-2$. 
\begin{lem}
There is an isomorphism
\[ K(2)_*K(2)\cong HH_*^{K(2)_*}(K(2)_*\tBP{2};K(2)_*K(2))\] 
and hence the $K(2)_*$-based Adams spectral sequence collapses with no room for hidden extensions and the natural map 
\[ K(2)_*K(2)\rightarrow K(2)_*THH(\tBP{2};K(2)) \]
is an isomorphism
\end{lem}
\begin{proof}
Since $K(2)_*\tBP{2}$ is flat over $K(2)_*$
\[ HH_*^{K(2)_*}(K(2)_*\tBP{2})\cong Tor^{K(2)_*\tBP{2}\otimes_{K(2)_*}K(2)_*\tBP{2}}_*(K(2)_*\tBP{2}; K(2)_*\tBP{2}).\]
Also, by \eqref{K(2) computation}, 
\[ 
\begin{array}{l}
Tor^{(K(2)_*\tBP{2})^{e}}_*(K(2)_*\tBP{2}; K(2)_*\tBP{2})\cong  \\
\bigotimes_{k\ge 1} Tor^{ K(2)_*[t_k]/(v_2t_k^{p^2}-v_2^{p^k}t_k ))^{e}}_*(K(2)_*[t_k]/(v_2t_k^{p^2}-v_2^{p^k}t_k ); K(2)_*[t_k]/(v_2t_k^{p^2}-v_2^{p^k}t_k ))\cong \\
\bigotimes_{k\ge 1; k|2}K(2)_*\otimes  HH_*^{K(2)_*}(K(2)_*[u_k]/(v_2u_k^{p^2}-v_2^{p^k}u_k ))\otimes \\
\bigotimes_{k\ge 1; (k+1)|2}K(2)_*\otimes  HH_*^{K(2)_*}(K(2)_*[w_k]/(v_2w_k^{p^2}-v_2^{p^k}w_k ))
\end{array}
\]
By Cartan-Eilenberg, for $k\ge0$ an odd integer
\[HH_*^{K(2)_*}(K(2)_*[w_k]/(v_2w_k^{p^2}-v_2^{p^k}w_k )\cong K(2)_*[w_k]/(v_2w_k^{p^2}-v_2^{p^k}w_k)\otimes_{K(2)_*} Tor^{K(2)_*K(2)_*[w_k]/(v_2w_k^{p^2}-v_2^{p^k}w_k }(K(2)_*,K(2)_*)\]
and by an elementary calculation, 
\[Tor^{K(2)_*K(2)_*[w_k]/(v_2w_k^{p^2}-v_2^{p^k}w_k }(K(2)_*,K(2)_*)\cong K(2)_*\]
and therefore 
\[HH_*^{K(2)_*}(K(2)_*[w_k]/(v_2w_k^{p^2}-v_2^{p^k}w_k )\cong K(2)_*[w_k]/(v_2w_k^{p^2}-v_2^{p^k}w_k). \]
Also, there is an isomorphism 
\[HH_*^{K(2)_*}(K(2)_*[u_k]/(v_2u_k^{p^2}-v_2^{p^k}u_k ))\cong K(2)_*\otimes HH_*(\mathbb{F}_p[u_k]/(u_k^{p^2}-u_k ))\]
and since 
\[\mathbb{F}_p[u_k]/(u_k^{p^2}-u_k  \]
is isomorphic as a $\mathbb{F}_p$-algebra to a product of finite field extensions of $\mathbb{F}_p$ (\textcolor{seagreen}{We should be more precise here.}) and since Hochschild homology commutes with limits and $HH_*(\mathbb{F}_{p^n})\cong \mathbb{F}_{p^n})$,
\[ HH_*(\mathbb{F}_p[u_k]/(u_k^{p^2}-u_k ))\cong \mathbb{F}_p[u_k]/(u_k^{p^2}-u_k ). \]
Putting this all together, we produce an isomorphism
\[ K(2)_*\tBP{2}\cong HH_*^{K(2)_*}(K(2)_*\tBP{2})\] 
and since 
\[ HH_*^{K(2)_*}(K(2)_*\tBP{2};K(2)_*K(2))\cong K(2)_*K(2)\otimes_{K(2)_*\tBP{2}} HH_*^{K(2)_*}(K(2)_*\tBP{2})\]
we produce the desired isomorphism 
\[K(2)_*K(2)\cong HH_*^{K(2)_*}(K(2)_*\tBP{2};K(2)_*K(2))\]
The B\"okstedt spectral sequence 
\[HH_*^{K(2)_*}(K(2)_*\tBP{2};K(2)_*K(2)) \Rightarrow  K(2)_*THH(\tBP{2};K(2)) \]
therefore collapses with no room for hidden extensions and hence the map 
\[ K(2)\rightarrow THH(\tBP{2};K(2)) \]
induces a $K(2)_*$-equivalence.
\end{proof}
\begin{cor}
The map $K(2)\rightarrow THH(\tBP{2};K(2))$ is a weak equivalence and therefore 
\[ THH_*(\tBP{2};k(2))\cong P(v_2)\otimes T \] 
where $T$ is a $v_2$-torsion $P(v_2)$-module. 
\end{cor}
\begin{proof}
Since the map $K(2)\rightarrow THH(\tBP{2};K(2))$ induces an isomorphism $K(2)_*K(2)\cong K(2)_*THH(\tBP{2};K(2))$, the $K(2)$-based Adams spectral sequence for $THH(\tBP{2};K(2))$ converges and collapses to the zero line and the map of $K(2)$-based Adams spectral sequences induces an isomorphism
\[ K(2)_*\rightarrow THH_*(\tBP{2};K(2)).\] 
Since we have a map that induces an isomorphism on homotopy groups the Whitehead theorem for spectra implies that the map $K(2)\rightarrow THH(\tBP{2};K(2))$ is a weak equivalence. 

Alternatively, we could compute $THH_*(\tBP{2};K(2))$ using the $v_2$-inverted classical Adams spectral sequence, which is equivalent to the Bockstein spectral sequence 
\[ THH_*(\tBP{2};H\mathbb{F}_p)[v_2^{\pm 1}]\Rightarrow THH_*(\tBP{2};K(2))\]
and by the computation we just did, we know that all the classes must die except those in $P(v_2^{\pm 1})$. There is also a map of spectral sequences 
\[ 
\xymatrix{
THH_*(\tBP{2};H\mathbb{F}_p)[v_2^{\pm 1}] \ar@{=>}[r] & THH_*(\tBP{2};K(2)) \\
THH_*(\tBP{2};H\mathbb{F}_p)[v_2]  \ar[u] \ar@{=>}[r] &THH_*(\tBP{2};k(2)) \ar[u] 
}
\] 
because $v_2^{-1}(-)$ is a localization. This implies that 
\[ THH_*(\tBP{2};k(2))\cong P(v_2)\otimes T \] 
and forces differentials in the bottom spectral sequence above.
\end{proof}
\begin{cor}
There are differentials $d_{r(n)}(\mu^{r(n)})=\lambda_{[n]}v_2^{r(n)}$
where $r(n)$ is $\dots$, and $\lambda_{[n]}$ is $\dots$.
\end{cor}
\textcolor{seagreen}{Finish the corollary above.}


\bibliographystyle{plain}
\bibliography{THHBP2}

\end{document}