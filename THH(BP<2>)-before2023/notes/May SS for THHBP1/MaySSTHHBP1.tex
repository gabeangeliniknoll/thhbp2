\documentclass[12pt]{amsart}
%\usepackage[urw-garamond]{mathdesign}

\usepackage[T1]{fontenc}


\let\circledS\undefined % here - PS


\usepackage{amsmath}
\usepackage{amsthm}
\usepackage{amssymb}
\usepackage{lscape,xcolor}
\usepackage{graphicx}
\usepackage{mathrsfs}
\usepackage{stmaryrd}
\usepackage{verbatim}
\usepackage{rotating}
\usepackage{tikz-cd}
\usepackage{amsrefs}
\usepackage{hyperref}
\usepackage{euscript}
\usepackage[colorinlistoftodos]{todonotes}

\usepackage{luasseq}
\usepackage{xcolor}
\definecolor{seagreen}{RGB}{46,139,87}
\definecolor{maroon}{RGB}{128,0,0}
\definecolor{darkviolet}{RGB}{148,0,211}
\definecolor{twelve}{RGB}{100,100,170}
\definecolor{thirteen}{RGB}{100,150,50}
\definecolor{fourteen}{RGB}{200,0,0}
\definecolor{fifteen}{RGB}{0,200,0}
\definecolor{sixteen}{RGB}{0,0,200}
\definecolor{seventeen}{RGB}{200,0,200}
\definecolor{eighteen}{RGB}{0,200,200}



%\parskip 0.7pc
%\parindent 0pt

\allowdisplaybreaks[1]

%%%%%%%%%%%%%%% Basic commands %%%%%%%%%%%%%%%%%%
\newcommand{\dotequiv}{\overset{\scriptstyle{\centerdot}}{\equiv}}
\newcommand{\nd}{\not\!|}
\newcommand{\mmod}{\! \sslash \!}

\newcommand{\mc}[1]{\mathcal{#1}}
\newcommand{\ull}[1]{\underline{#1}}
\newcommand{\mb}[1]{\mathbb{#1}}
\newcommand{\mr}[1]{\mathrm{#1}}
\newcommand{\mbf}[1]{\mathbf{#1}}
\newcommand{\mit}[1]{\mathit{#1}}
\newcommand{\mf}[1]{\mathfrak{#1}}
\newcommand{\ms}[1]{\mathscr{#1}}
\newcommand{\abs}[1]{\lvert #1 \rvert}
\newcommand{\norm}[1]{\lVert #1 \rVert}
\newcommand{\bra}[1]{\langle #1 \rangle}
\newcommand{\br}[1]{\overline{#1}}
\newcommand{\brr}[1]{\overline{\overline{#1}}}
\newcommand{\td}[1]{\widetilde{#1}}
\newcommand{\tdd}[1]{\widetilde{\widetilde{#1}}}
\newcommand{\Z}{\mathbb{Z}}
\newcommand{\R}{\mathbb{R}}
\newcommand{\C}{\mathbb{C}}
\newcommand{\Q}{\mathbb{Q}}
\newcommand{\W}{\mathbb{W}}
\newcommand{\F}{\mathbb{F}}
\newcommand{\G}{\mathbb{G}}
\newcommand{\MS}{\mathbb{S}}
\newcommand{\PP}{\mathbb{P}}

\newcommand{\euscr}[1]{\EuScript{#1}}

%%%%%%%%%%%%%%%%% Spectra %%%%%%%%%%%%%%%

\newcommand{\bbS}{\mathbb{S}}
\newcommand{\tBP}[1]{BP\bra{#1}}
\newcommand{\AF}{\mr{AF}}
\newcommand{\TAF}{\mathrm{TAF}}
\newcommand{\TMF}{\mathrm{TMF}}
\newcommand{\Tmf}{\mathrm{Tmf}}
\newcommand{\tmf}{\mathrm{tmf}}
\newcommand{\bo}{\mathrm{bo}}
\newcommand{\bsp}{\mathrm{bsp}}
\newcommand{\HZ}{\mr{H}\Z}
\def \HF2{\mr{H}\F_2}
\newcommand{\bu}{\mr{bu}}
\newcommand{\MU}{\mr{MU}}
\newcommand{\KU}{\mr{KU}}
\newcommand{\KO}{\mr{KO}}
\newcommand{\EO}{\mr{EO}}
\newcommand{\BP}{\mr{BP}}
\newcommand{\K}{\mr{K}}

%%%%%%%%%%%%%%% Operators %%%%%%%%%%%%%%

\DeclareMathOperator{\Ext}{Ext}
\DeclareMathOperator{\aut}{Aut}
\DeclareMathOperator{\im}{im}
\DeclareMathOperator{\Sta}{Sta}
\DeclareMathOperator{\Map}{Map}
\DeclareMathOperator*{\holim}{holim}
\DeclareMathOperator*{\hocolim}{hocolim}
\DeclareMathOperator*{\colim}{colim}
\DeclareMathOperator*{\Tot}{Tot}
\DeclareMathOperator{\Spf}{Spf}
\DeclareMathOperator{\Aut}{Aut}
\DeclareMathOperator{\Spec}{Spec}
\DeclareMathOperator{\Proj}{Proj}
\DeclareMathOperator{\THH}{THH}

\DeclareMathOperator{\sq}{Sq}
\newcommand{\xib}{{\bar{\xi}}}
\newcommand{\s}{\wedge}
\newcommand{\Si}{\Sigma}
\newcommand\Floor[1]{\lfloor#1\rfloor}

%%%%%%%%%%%%% Steenrod Algebra & Brown-Gitler Modules %%%%%%%%%%%

\newcommand{\A}{\ms{A}}
\newcommand{\sE}{\ms{E}}
\newcommand{\HZu}{\ull{\HZ}}
\newcommand{\bou}{\ull{\bo}}
\newcommand{\tmfu}{\ull{\tmf}}
\newcommand{\tBPu}[1]{\ull{\tBP{#1}}}
\newcommand{\buu}{\ull{\bu}}
\def \AA0{\br{A \mmod A(0)}_*}
\def \AA2{A\mmod A(2)_*}
\def \AE2{(A\mmod E(2))_*}
\renewcommand{\AE}[1]{(A\mmod E(#1))_*}
\DeclareMathOperator{\wt}{\mathrm{wt}}
\def \E2E1{(E(2)\mmod E(1))_*}
\newcommand{\otau}{\overline{\tau}}



%%%%%%%%%%%%%%%% Categories %%%%%%%%%%%%%

\newcommand{\Top}{\mathsf{Top}}
\newcommand{\Operad}{\mathsf{Operad}}
\newcommand{\Alg}{\mathsf{Alg}}
\newcommand{\Monad}{\mathsf{Monad}}
\newcommand{\Set}{\mathsf{Set}}
\newcommand{\sSet}{\mathsf{sSet}}
\newcommand{\Man}{\mathsf{Man}}
\newcommand{\Presheaf}{\mathsf{Presheaf}}
\newcommand{\Fun}{\mathsf{Fun}}
\newcommand{\Grpd}{\mathsf{Grpd}}
\newcommand{\Sp}{\mathsf{Sp}}
\newcommand{\Aff}{\Mathsf{Aff}}
\newcommand{\CAlg}{\mathsf{CAlg}}
\newcommand{\Mod}{\mathsf{Mod}}
\newcommand{\op}{\mathsf{op}}
\newcommand{\QCoh}{\mathsf{QCoh}}



%%%%%%%%%%%%%%% Homological Algebra %%%%%%%

\newcommand{\cone}[1]{\mathrm{cone}\left(#1\right)}

%%%%%%%%% THH %%%%%%%%%%%%

\newcommand{\tilmu}{\tilde{\mu}}
\newcommand{\MayE}{\mbox{}^{May}E}


%%%%%%% for numbered theorems %%%%%%%%%
 \newtheorem{thm}[equation]{Theorem}
 \newtheorem{cor}[equation]{Corollary}
 \newtheorem{lem}[equation]{Lemma}
 \newtheorem{prop}[equation]{Proposition}
 \newtheorem{obs}[equation]{Observation}
  \newtheorem{rem}[equation]{Remark}
 
 \newtheorem*{thm*}{Theorem}
 \newtheorem*{cor*}{Corollary}
 \newtheorem*{lem*}{Lemma}
 \newtheorem*{prop*}{Proposition}
  \newtheorem*{not*}{Notation}

 
 \theoremstyle{definition}
 \newtheorem{defn}[equation]{Definition}
 \newtheorem{ex}[equation]{Example}
 \newtheorem{exs}[equation]{Examples}
 \newtheorem{rmk}[equation]{Remark}
\newtheorem{claim}[equation]{Claim}
 \newtheorem{question}[equation]{Question}
 \newtheorem{conjecture}[equation]{Conjecture}
%%%%%%%%%%%%%%%%%%%%%%%%%%%%%%%%%%%%%%%%

\newtheorem*{defn*}{Definition}
\newtheorem*{ex*}{Example}
\newtheorem*{exs*}{Examples}
\newtheorem*{rmk*}{Remark}
\newtheorem*{claim*}{Claim}
\newtheorem*{conventions}{Conventions}
\numberwithin{equation}{section}
\numberwithin{figure}{section}



\title[Hochschild-May spectral sequence for $THH(\ell)$]{Hochschild-May spectral sequence for topological Hochschild homology of the Adams summand}
\author{G. Angelini-Knoll}\address{Michigan State University, East Lansing}{\email{angelini@math.msu.edu}
\author{D.~ Culver}\address{University of Illinois, Urbana-Champaign}\email{dculver@illinois.edu}

\begin{document}

\maketitle



\begin{abstract}
We give a new computation of topological Hochschild homology the Adams summand, originally due to \cite{AHL}, using the Hochschild-May spectral sequence developed by the first author and Andrew Salch in \cite{AS18}. 
\end{abstract}

\tableofcontents

\section{Introduction}
The goal of this note is to compute $THH_*(\ell)$ using the Hochschild-May spectral sequence developed by the first author and Andrew Salch in \cite{AS18}. 
\section{THH of $\ell$ with $H\F_p$, $H\Z_p$, and $k(1)$ coefficients}
We first compute the Hochschild-May spectral sequences for $THH_*(\ell;H\F_p)$, $THH_*(\ell;H\Z_p)$ and $THH_*(\ell;k(1))$ using the Whitehead filtration. Throughout we write $\ell$ for the $p$-completion of the Adams summand.

We first recall that B\"okstedt computed 
\[ THH_n(\Z_p)\cong 
	\begin{cases} 
		\Z & n=0 \\ 
		\Z/p^{\nu_p(m)}\{\gamma_{m}\} & n=2m-1\\
		0 & \text{oterwise}
	\end{cases}
\]
where $\nu_p(k)$ is the $p$-adic valuation of $k$.

\begin{lem}
The Hochschild-May spectral sequence 
\[ E_2^{*,*}=THH_*(H\pi_*\ell;H\F_p)\Rightarrow THH_*(\ell;H\F_p)\]
has $E_2$-page 
\[ E_2^{*,*}\cong THH_*(H\Z_p;H\F_p)\otimes E(\sigma v_1)\]
where 
\[ THH_*(H\Z_p;H\F_p)\cong E(\lambda_1)\otimes P(\mu_1)\]
and 
the differentials are generated by the differential 
\[ d_{1}(\mu_1)=\sigma v_1 \]
\end{lem}
\begin{proof}
We first observe that there is an equivalence $H\pi_*\ell\simeq H\Z\wedge S[v_1]$ as $A_{\infty}$-ring spectra where $S[v_1]$ is the free $A_{\infty}$-algebra $\bigvee_{i\ge 0} S^{(2p-2)i}$. The first statement then follows by rearranging colimits 
\[
	\begin{array}{rcl}
	H\F_p\wedge_{H\Z_p\wedge S[v_1]} THH(H\Z_p\wedge S[v_1])&\simeq &\\
	H\F_p\wedge_{H\Z_p\wedge S[v_1]} \left (THH(H\Z_p)\wedge THH(S[v_1]) \right )&\simeq  \\
	THH(H\Z_p;\F_p)\wedge_{H\F_p} H\F_p\wedge THH(S[v_1],S), & &
 	\end{array}
\]
applying the K\"unneth isomorphism, and then computing 
\[ H_*THH(S[v_1],S)\cong E(\sigma v_1)\]
using the B\"okstedt spectral sequence, which clearly collapses. 
We know that $THH_{2p}(\ell,H\F_p)\cong 0$, which forces the differential $d_{1}(\mu_1)=\sigma v_1$ and this is the only possible differential whose source is a generator in this two line spectral sequence. 
\end{proof}

\begin{lem}
The Hochschild-May spectral sequence
\[ E_2^{*,*}=THH_*(H\pi_*\ell;H\Z_p)\Rightarrow THH_*(\ell;H\Z_p)\]
has $E_2$-page
\[E_2^{*,*}= THH_*(\Z_p)\otimes_{\Z_p} E_{\Z_p}(\sigma v_1)\]
and differentials 
\[ d_1(\gamma_{pk})=p^{\alpha(k)} \sigma v_1 \gamma_{p(k-1)}\]
where $\alpha(k)=\min\{0,\nu_p(k-1)-\nu_p(k)\}$.

The classes $a_i$ are detected by $p\gamma_{ip^2}$ and the classes $b_i$ are detected by $\gamma_{ip^2}\sigma v_1$, up to multiplication by a unit. There is also a hidden extension 
\[ p\gamma_p=\sigma v_1.\]
\end{lem}
\begin{proof}
The identification of the $E_2$-page is exactly the same argument where the existence of the B\"okstedt spectral sequence computing $H\Z_*THH(S[v_1],S)$, in this case, follows because $H\Z_*S[v_1]$ is free over $H\Z_*$. The only possible differentials are the ones stated and these must occur in order to produce the known answer due to \cite{AHL}. The detection results follow by comparison with the results of \cite{AHL}. We point out that the hidden extension could be derived explicitly using the formula 
\[ p\lambda_1=\sigma v_1\]
due to \cite{Rog19} since $\gamma_p=\lambda_1$. 
\end{proof}
\begin{lem}
The Hochschild-May spectral sequence
\[ E_2^{*,*}=THH_*(H\pi_*\ell;H\pi_*k(1))\Rightarrow THH_*(\ell;k(1))\]
has $E_2$-page
\[ THH_*(H\Z_p;H\F_p)\otimes P(v_1)\otimes E(\sigma v_1).\]
The first nontrivial differential is 
\[ d_{1}(\mu_1)=\sigma v_1 \]
and the remaining differentials are determined by the differentials in the Bockstein spectral sequence 
\[ THH_*(\ell;H\F_p)[v_1]\Rightarrow THH_*(\ell;k(1))\]
since the $E_3$-page of the spectral sequence is isomorphic to the associated graded of a filtration on $THH_*(\ell;H\F_p)$. In particular, the differentials are 
\[ d_{r'(n)}(\mu_1^{r'(n)})=v_1^{r'(n)-\epsilon(n)}\lambda_n\]
where $\epsilon(n)=1$ if $n$ is even and zero otherwise and $r'(n)$ is defined recursively by $r'(0)=1,$ $r'(1)=p$ and 
\[ r'(n)=p^n+r'(n-2)\]
for $n\ge 0$.
The elements $\lambda_n$ are also defined recursively by $\lambda_n=\lambda_{n-2}\mu_1^{p^{n-2}(p-1)}$ where $\lambda_0=\sigma v_1$.
\end{lem}
\begin{proof}
Again, the argument for computing the $E_2$-page is essentially the same. The first differential can be surmised from the map of Hochschild-May spectral spectral sequences
\[ THH_*(H\pi_*\ell;H\pi_*k(1))\to THH_*(H\pi_*\ell;H\F_p)\]
which sends $v_1$ to zero and is a bijection on all elements that are not $v_1$-divisible. We write $E_{\infty}^{*,*}(\ell;H\F_p)$ for the $E_{\infty}$-page of the latter spectral sequence. Then we see that there is an isomorphism 
\[ E_3^{*,*}\cong E_{\infty}^{*,*}(\ell;H\F_p)\otimes P(v_1).\]
we now recall that, by \cite{MS}, the differentials in the Bockstein spectral sequence are generated by 
\[ d_{r(n)}(\mu_2^{r(n)})=v_1^{r(n)}\lambda_n.\]
so by identifying $\mu_2=\mu_1^p$ and $\lambda_2=\lambda_1\mu_1^{p-1}$ we get the desired result.
\end{proof}
\bibliographystyle{alpha}
\bibliography{THH}
\end{document}