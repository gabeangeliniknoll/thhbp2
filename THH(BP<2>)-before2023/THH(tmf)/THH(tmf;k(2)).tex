
\documentclass[12pt]{amsart}

\usepackage[urw-garamond]{mathdesign}
\usepackage[T1]{fontenc}

\let\circledS\undefined % here - PS


\usepackage{amsmath}
\usepackage{amsthm}
\usepackage{amssymb}
\usepackage{lscape,xcolor}
\usepackage{graphicx}
\usepackage{mathrsfs}
\usepackage{stmaryrd}
\usepackage{verbatim}
\usepackage{rotating}
\usepackage{tikz-cd}
\usepackage{amsrefs}
\usepackage{hyperref}
\usepackage{euscript}
\usepackage[colorinlistoftodos]{todonotes}


\usepackage[sc]{mathpazo}
\linespread{1.05}         % Palatino needs more leading (space between lines)
\usepackage[T1]{fontenc}

\usepackage[OT2,T1]{fontenc}
\newcommand\textcyr[1]{{\fontencoding{OT2}\fontfamily{wncyr}\selectfont #1}}


\usepackage{luasseq}
\usepackage{xcolor}
\definecolor{seagreen}{RGB}{46,139,87}
\definecolor{maroon}{RGB}{128,0,0}
\definecolor{darkviolet}{RGB}{148,0,211}
\definecolor{twelve}{RGB}{100,100,170}
\definecolor{thirteen}{RGB}{100,150,50}
\definecolor{fourteen}{RGB}{200,0,0}
\definecolor{fifteen}{RGB}{0,200,0}
\definecolor{sixteen}{RGB}{0,0,200}
\definecolor{seventeen}{RGB}{200,0,200}
\definecolor{eighteen}{RGB}{0,200,200}



%\parskip 0.7pc
%\parindent 0pt

\allowdisplaybreaks[1]

%%%%%%%%%%%%%%% Basic commands %%%%%%%%%%%%%%%%%%
\newcommand{\dotequiv}{\overset{\scriptstyle{\centerdot}}{\equiv}}
\newcommand{\nd}{\not\!|}
\newcommand{\mmod}{\! \sslash \!}

\newcommand{\mc}[1]{\mathcal{#1}}
\newcommand{\ull}[1]{\underline{#1}}
\newcommand{\mb}[1]{\mathbb{#1}}
\newcommand{\mr}[1]{\mathrm{#1}}
\newcommand{\mbf}[1]{\mathbf{#1}}
\newcommand{\mit}[1]{\mathit{#1}}
\newcommand{\mf}[1]{\mathfrak{#1}}
\newcommand{\ms}[1]{\mathscr{#1}}
\newcommand{\abs}[1]{\lvert #1 \rvert}
\newcommand{\norm}[1]{\lVert #1 \rVert}
\newcommand{\bra}[1]{\langle #1 \rangle}
\newcommand{\br}[1]{\overline{#1}}
\newcommand{\brr}[1]{\overline{\overline{#1}}}
\newcommand{\td}[1]{\widetilde{#1}}
\newcommand{\tdd}[1]{\widetilde{\widetilde{#1}}}
\newcommand{\Z}{\mathbb{Z}}
\newcommand{\R}{\mathbb{R}}
\newcommand{\C}{\mathbb{C}}
\newcommand{\Q}{\mathbb{Q}}
\newcommand{\W}{\mathbb{W}}
\newcommand{\F}{\mathbb{F}}
\newcommand{\G}{\mathbb{G}}
\newcommand{\MS}{\mathbb{S}}
\newcommand{\PP}{\mathbb{P}}

\newcommand{\euscr}[1]{\EuScript{#1}}

%%%%%%%%%%%%%%%%% Spectra %%%%%%%%%%%%%%%

\newcommand{\tBP}[1]{BP\bra{#1}}
\newcommand{\AF}{\mr{AF}}
\newcommand{\TAF}{\mathrm{TAF}}
\newcommand{\TMF}{\mathrm{TMF}}
\newcommand{\Tmf}{\mathrm{Tmf}}
\newcommand{\tmf}{\mathrm{tmf}}
\newcommand{\bo}{\mathrm{bo}}
\newcommand{\bsp}{\mathrm{bsp}}
\newcommand{\HZ}{\mr{H}\Z}
\def \HF2{\mr{H}\F_2}
\newcommand{\bu}{\mr{bu}}
\newcommand{\MU}{\mr{MU}}
\newcommand{\KU}{\mr{KU}}
\newcommand{\KO}{\mr{KO}}
\newcommand{\EO}{\mr{EO}}
\newcommand{\BP}{\mr{BP}}
\newcommand{\K}{\mr{K}}
\newcommand{\THH}{\mathrm{THH}}

%%%%%%%%%%%%%%% Operators %%%%%%%%%%%%%%

\DeclareMathOperator{\Ext}{Ext}
\DeclareMathOperator{\aut}{Aut}
\DeclareMathOperator{\im}{im}
\DeclareMathOperator{\Sta}{Sta}
\DeclareMathOperator{\Map}{Map}
\DeclareMathOperator*{\holim}{holim}
\DeclareMathOperator*{\hocolim}{hocolim}
\DeclareMathOperator*{\colim}{colim}
\DeclareMathOperator*{\Tot}{Tot}
\DeclareMathOperator{\Spf}{Spf}
\DeclareMathOperator{\Aut}{Aut}
\DeclareMathOperator{\Spec}{Spec}
\DeclareMathOperator{\Proj}{Proj}

\DeclareMathOperator{\sq}{Sq}
\newcommand{\xib}{{\bar{\xi}}}
\newcommand{\s}{\wedge}
\newcommand{\Si}{\Sigma}
\newcommand\floor[1]{\lfloor#1\rfloor}

%%%%%%%%%%%%% Steenrod Algebra & Brown-Gitler Modules %%%%%%%%%%%

\newcommand{\A}{\ms{A}}
\newcommand{\sE}{\ms{E}}
\newcommand{\HZu}{\ull{\HZ}}
\newcommand{\bou}{\ull{\bo}}
\newcommand{\tmfu}{\ull{\tmf}}
\newcommand{\tBPu}[1]{\ull{\tBP{#1}}}
\newcommand{\buu}{\ull{\bu}}
\def \AA0{\br{A \mmod A(0)}_*}
\def \AA2{A\mmod A(2)_*}
\def \AE2{(A\mmod E(2))_*}
\renewcommand{\AE}[1]{(A\mmod E(#1))_*}
\DeclareMathOperator{\wt}{\mathrm{wt}}
\def \E2E1{(E(2)\mmod E(1))_*}



%%%%%%%%%%%%%%%% Categories %%%%%%%%%%%%%

\newcommand{\Top}{\mathsf{Top}}
\newcommand{\Operad}{\mathsf{Operad}}
\newcommand{\Alg}{\mathsf{Alg}}
\newcommand{\Monad}{\mathsf{Monad}}
\newcommand{\Set}{\mathsf{Set}}
\newcommand{\sSet}{\mathsf{sSet}}
\newcommand{\Man}{\mathsf{Man}}
\newcommand{\Presheaf}{\mathsf{Presheaf}}
\newcommand{\Fun}{\mathsf{Fun}}
\newcommand{\Grpd}{\mathsf{Grpd}}
\newcommand{\Mell}{\mathscr{M}_{\text{ell}}}
\newcommand{\M}{\mathscr{M}}


%%%%%%%%%%%%%%% Homological Algebra %%%%%%%

\newcommand{\cone}[1]{\mathrm{cone}\left(#1\right)}


%%%%%%% for numbered theorems %%%%%%%%%
 \newtheorem{thm}[equation]{Theorem}
 \newtheorem{cor}[equation]{Corollary}
 \newtheorem{lem}[equation]{Lemma}
 \newtheorem{prop}[equation]{Proposition}
 \newtheorem{obs}[equation]{Observation}
  \newtheorem{rem}[equation]{Remark}
 
 \newtheorem*{thm*}{Theorem}
 \newtheorem*{cor*}{Corollary}
 \newtheorem*{lem*}{Lemma}
 \newtheorem*{prop*}{Proposition}
  \newtheorem*{not*}{Notation}

 
 \theoremstyle{definition}
 \newtheorem{defn}[equation]{Definition}
 \newtheorem{ex}[equation]{Example}
 \newtheorem{exs}[equation]{Examples}
 \newtheorem{rmk}[equation]{Remark}
\newtheorem{claim}[equation]{Claim}
 \newtheorem{question}[equation]{Question}
 \newtheorem{conjecture}[equation]{Conjecture}
%%%%%%%%%%%%%%%%%%%%%%%%%%%%%%%%%%%%%%%%

\newtheorem*{defn*}{Definition}
\newtheorem*{ex*}{Example}
\newtheorem*{exs*}{Examples}
\newtheorem*{rmk*}{Remark}
\newtheorem*{claim*}{Claim}
\newtheorem*{conventions}{Conventions}
\numberwithin{equation}{section}
\numberwithin{figure}{section}





\author{Gabe Angelini-Knoll}\address{Michigan State University, Lansing}
	\email{gabe@gabe@gmail.com}
	\author{D.~ Culver}\address{University of Illinois, Urbana-champaign}\email{dculver@illinois.edu}
	
	\title{Topological Hochschild homology of $\tmf$ with coefficients in $k(2)$}

\begin{document}

\maketitle

%\tableofcontents

\section{Introduction} 

The purpose of this short document is to carry out a computation of $\THH(\tmf; k(2))$ at the prime $p=2$. We take advantage of the fact that Bhattacharya-Egger have recently constructed a finite spectrum $Z$ with the property that it has a $v_2^1$-self map and gives an equivalence
\[
\tmf\wedge Z\simeq k(2),
\]
where $k(2)$ denotes the connective second Morava K-theory.

\section{Calculations}

Let's start the calculation. We begin by computing the $K(2)$-homology of $\tmf$. After that we run the $v_2$-Bockstein spectral sequence. 

\subsection{$K(2)_*\tmf$}

There are two steps to this calculation. We start by computing the Adams spectral sequence for $k(2)_*\tmf$, and then invert $v_2$. This gives an associated graded calculation of $K(2)_*\tmf$. Then, to resolve hidden extensions, we determine the map
\[
K(2)_*\tmf\to K(2)_*\tBP{2},
\]
and use the known calculation of $K(2)_*\tBP{2}$.

The Adams spectral sequence for $k(2)\wedge \tmf$ takes the form 
\[
\Ext_{A_*}(k(2)\wedge \tmf)\implies k(2)_*\tmf.
\]

A change-of-rings isomorphism allows us to express the $E_2$-page as 
\[
E_2\cong \Ext_{E(Q_2)}(H_*\tmf)\cong (P(v_2)\otimes M_*(\tmf;Q_2))\oplus (v_2\text{-torsion}).
\]
Recall that the mod 2 homology of $\tmf$ is 
\[
H_*(\tmf)\cong \AA2\cong P(\zeta_1^8, \zeta_2^4, \zeta_3^2, \zeta_4, \ldots).
\]
The $Q_2$-action on $H_*\tmf$ is 
\[
Q_2(\zeta_k) = \zeta_{k-3}^8.
\]

Thus the Margolis homology is 
\[
M_*(\tmf;Q_2)\cong P(\zeta_2^4, \zeta_3^2, \zeta_4^2, \zeta_5^2, \ldots)/(\zeta_2^8, \zeta_3^8, \zeta_4^8, \ldots).
\]
It follows from this that the Adams spectral sequence collapses immediately. 

Inverting $v_2$ kills the $v_2$-torsion, and so we get that 
\[
v_2^{-1}\Ext_{A_*}(k(2)\wedge \tmf)\cong K(2)_*\otimes P(\zeta_2^4, \zeta_3^2, \zeta_4^2, \zeta_5^2, \ldots)/(\zeta_2^8, \zeta_3^8, \zeta_4^8, \ldots).
\]

Now onto the hidden extensions. First, note that there is a map of $E_\infty$-rings\footnote{This map is induced by taking global sections associated to the map of moduli stacks $\M_1(3)\to \Mell$ from the moduli of elliptic curves with level $\Gamma_1(3)$ to the moduli of all elliptic curves.}
\[
\tmf\to \tmf_1(3).
\]
A similar analysis with the ASS shows that the $E_\infty$-page for $K(2)\wedge \tmf$ is
\[
K(2)_*\otimes P(\zeta_1^2, \zeta_2^2, \zeta_3^2, \ldots)/(\zeta_1^8, \zeta_2^8, \zeta_3^8, \ldots).
\]
The morphism from $\tmf$ to $\tmf_1(3)$ induces the obvious map on $v_2$-inverted Ext groups. Thus we know the morphism
\[
K(2)_*\tmf\to K(2)_*\tmf_1(3)
\]
up to associated graded.

We also need to use the fact that $\tmf_1(3)$ is a \emph{form} of $\tBP{2}$: that is there is (canonical?) equivalence 
\[
\tBP{2}\simeq \tmf_1(3)
\]
after completing at $2$. Since localizing with respect to $K(2)$ also $p$-completes, we have that 
\[
L_{K(2)}\tBP{2}\simeq L_{K(2)}\tmf_1(3),
\]
and hence 
\[
K(2)_*(\tBP{2})\cong K(2)_*(\tmf_1(3)).
\]
We need to recall the following computation. 

\begin{thm}
		There is an isomorphism of graded rings
	\[
	K(2)_*\tBP{2}\cong K(2)_*[t_1, t_2, \ldots ]/(v_2t_k^{4}-v_2^{2^k}t_k\mid k\geq 1)
	\]
\end{thm}

We relate this calculation to the one we obtained through the Adams spectral sequence. First, note there is a canonical map of commutative ring spectra
\[
BP\to \tBP{2},
\]
which leads to a homomorphism
\[
BP_*BP\to H_*\tmf_1(3)
\]
sending $t_i$ to $\zeta_i^2$. Since $\zeta_i^2$ is sent to $\zeta_i^2$ under 
\[
v_2^{-1}\Ext_A(k(2)\wedge \tmf)\to v_2^{-1}\Ext_A(k(2)\wedge \tmf_1(3)),
\]
we obtain the following. 

\begin{prop}
	The $K(2)$-homology of $\tmf$ is given by 
	\[
	K(2)_*\tmf\cong K(2)_*[t_2^2, t_3, t_4, \ldots]/()
	\]\todo{I am not sure what the relation on $t_2^2$ is supposed to be...}
\end{prop}

\bibliographystyle{plain}
\bibliography{THH}

\end{document}