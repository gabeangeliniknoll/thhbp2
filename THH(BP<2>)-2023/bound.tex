\section{Bounding Hochschild homology of $\BP\langle n\rangle$}
The goal of this section is to use the cosimplicial descent spectral sequence from work of \cite{DR18} to produce a useful upper bound on $\THH_{*}(\BP\langle n\rangle)$. 
\begin{definition}
Let $C^{\bullet}(A/B)$ denote the cosimplicial cobar complex with $q$-simplices $C^{q}(A/B)=A^{\otimes_{B}q+1}$.
\end{definition}

First, we need a lemma. 

\begin{lemma}\label{cobar lemma}
Let $n\ge 1$. There is an isomorphism rings
\[\rE_{2}^{*,*}\cong \Tor^{\pi_{*}\BP\langle n-1\rangle \wedge \BP\langle n-1\rangle }(\BP \langle n-1\rangle,\BP \langle n-1\rangle)\otimes \Gamma \{ \sigma^{2}v_{n}^{(j)}: 1\le j\le q\}\otimes \Lambda (\sigma v_{1}^{(j)} : 1\le j\le q ) 
\]
where 
\[\rE_{2}^{*,*}=\pi_{*}\left ( \pi_{*}H\pi_{*}\THH(\BP\langle n-1\rangle)^{\wedge_{H\pi_{*}\THH(\BP\langle n\rangle)}q+1} \right )\]
is the $\rE_{2}$-term of the multiplicative K\"unneth spectral sequence
\[ \rE_{2}^{*,*}\implies \THH_{*}(\BP\langle n-1\rangle^{\wedge q+1})\,.\]
\end{lemma}
\begin{proof}
When $q=0$, then $\THH(\BP\langle n-1\rangle^{\wedge_{\BP\langle n\rangle}q+1})=\THH(\BP\langle n-1\rangle)$. 
We first compute 
$\pi_{*}(\BP\langle n-1\rangle\otimes_{\BP\langle n\rangle}\BP\langle n-1\rangle)$ by a K\"unneth spectral sequence. 
The $E_{2}$-term is $\BP\langle n-1\rangle_{*}\otimes \Lambda (\sigma v_{n})$ so it is concentrated in K\"unneth filtration $[0,1]$ and therefore the spectral sequence collapses because the targets of all differentials are zero groups. 
We then use the equivalence 
\begin{align*}
A\wedge_{B}A\wedge_{B}A\simeq (A\wedge_{B}A)\wedge_{A}(A\wedge_{B}A)
\end{align*}
where $A=\BP\langle n-1\rangle$ and $B=\BP\langle n\rangle$ and the fact that $\pi_{*}(\BP\langle n-1\rangle\wedge_{\BP\langle n\rangle}\BP\langle n-1\rangle)$ is free as a $\BP\langle n-1\rangle_{*}$-module to inductively determine from the K\"unneth spectral sequence that 
\[ \pi_{*}(\BP \langle n-1\rangle ^{\wedge_{\BP\langle n\rangle}q+1})\cong \BP\langle n-1\rangle_{*}\otimes \Lambda (\sigma v_{n}^{(1)},\dots ,\sigma v_{n}^{(q)})\,.\]
By obstruction theory, we determine that $\BP \langle n-1\rangle ^{\wedge_{\BP\langle n\rangle}q+1}$ is the smash product of square zero extensions 
\[ \left ( BP\langle n-1\rangle \vee \Sigma^{2p-1} BP\langle n-1\rangle \right )^{\wedge_{ BP\langle n-1\rangle}q} \,.\]
Consequently, we determine that 
\[ \pi_{*}\left ( \BP \langle n-1\rangle ^{\wedge_{\BP\langle n\rangle}q+1} \right )^{\wedge 2}\cong \pi_{*}(\BP \langle n-1\rangle^{\wedge 2})\otimes \Lambda (\sigma v_{n}^{(1)},\dots ,\sigma v_{n}^{(q)})^{\wedge 2} \,.\]
We then note that the K\"unneth spectral sequence sequence computing 
\[\THH_{*}(\BP \langle n-1\rangle^{\wedge_{\BP\langle n\rangle} q+1} )\] 
has $\rE_{2}$-term
\[ \Tor^{\pi_{*}\BP\langle n-1\rangle \wedge \BP\langle n-1}(\BP \langle n-1\rangle,\BP \langle n-1\rangle)\otimes \Gamma (\sigma^{2} v_{n}^{(1)},\dots ,\sigma^{2}v_{n}^{(q)})\otimes \Lambda (\sigma v_{n}^{(1)},\dots ,\sigma v_{n}^{(q)})\]
and that is exactly what we describe in the statement of the lemma. 
\end{proof}

\begin{proposition}\label{induction spectral sequence}
There is an equivalence 
\[ 
	\THH(\BP\langle n \rangle,\BP\langle n-1\rangle)\simeq \Tot \left ( \THH(C^{\bullet}(\BP\langle n-1\rangle/\BP\langle n\rangle,\BP\langle n-1\rangle)) \right ).
\]
Consequently, there is a spectral sequence 
\[ 
	\pi_{t-s}\lim_{\Delta} \Tot H\pi_s\THH(C^{\bullet}(\BP\langle n-1\rangle/\BP\langle n\rangle,\BP\langle n-1\rangle))\implies \pi_{t-s}\THH(\BP\langle n\rangle;\BP\langle n-1\rangle)
\]
associated to the filtration 
\begin{equation}\label{Tot-filtration}
	\lim \Tot \tau_{\ge s}\THH(C^{\bullet}(\BP\langle n-1\rangle/\BP\langle n\rangle ;\BP\langle n-1\rangle))\,.
\end{equation}
The $E_2$-term is
\[ \THH_*(BP\langle n-1\rangle)\otimes_{\mathbb{Z}_{(p)}} \Lambda_{\mathbb{Z}_{(p)}} (\sigma v_n) \,.\]
Consequently,
\[ |\THH_{t}(BP\langle n \rangle;\BP\langle n-1)|\le |\THH_{t}(BP\langle n-1 \rangle)| + |\THH_{t-2p^{n}+1}(BP\langle n-1 \rangle)|\]
\end{proposition}
\begin{proof}
Since $\mathrm{BP}\langle n\rangle\to \mathrm{BP}\langle n-1\rangle$ is an isomorphism on $\pi_i$ for $i=0,1$ the first statement follows directly from \cite[Theorem 3.7]{DR18}. 
The second statement follows from \cite[Remark 3.7]{GIKR22} which identifies the filtration \eqref{Tot-filtration} with the d\'ecalage (cf. \cite[pp. 21]{Del71}) of the filtration whose associated graded is the $E_{1}$-term of the Bousfield--Kan spectral sequence. 

It therefore suffices to compute the $E_{2}$-term, which is the cohomology of the Hopf algebroid $(\THH_{*}(\BP\langle n-1\rangle), \THH_{*}(\BP\langle n-1\rangle)\otimes \Gamma \{ \sigma^{2}v_{n} \} )$ by Lemma \ref{cobar lemma}. 
We note from the proof of Lemma \ref{cobar lemma} that this Hopf algebroid is the tensor product of the Hopf algebroids 
$(\THH_{*}(\BP\langle n-1\rangle), \THH_{*}(\BP\langle n-1\rangle))$ and $(\mathbb{Z}_{(p)},\Gamma_{\mathbb{Z}_{(p)}} \{ \sigma^{2}v_{n})\} $. 
Consequently, the cohomology of this Hopf algebroid is $\THH_{*}(\BP\langle n-1\rangle )\otimes \Lambda (\sigma v_{n})$ as desired. 
\end{proof}

\begin{example}
We consider the case $n=0$. Then there is a spectral sequence 
\[ 
\THH_{*}(H\mathbb{F}_{p})\otimes \Lambda (\sigma v_{0})\implies \THH_{*}(\mathbb{Z}_{(p};\mathbb{F}_{p}) \,.
\]
This spectral sequence has a differential $d_{1}(\mu)=\sigma v_{0}$ and no further differentials except those generated by the Leibniz rule yielding the known answer $\mathbb{F}_p[\mu^p]\langle \sigma v_0 \mu^{p-1}\rangle$. 
\end{example}

\subsection{The Bockstein spectral sequences}

Associated to the square 
\[
\begin{tikzcd}
\BP\langle 1\rangle \ar[r] \ar[d] & H\mathbb{Z}_{(p)} \ar[d] \\ 
k(1) \ar[r] & H\mathbb{F}_p
\end{tikzcd}
\]
there is a square of Bockstein spectral sequences \[
\begin{tikzcd}
\THH(\BP\langle 2\rangle ;\mathbb{F}_p)[v_0,v_1]\arrow[Rightarrow]{r} \arrow[Rightarrow]{d} &  \THH(\BP\langle 2\rangle ;\mathbb{Z}_{(p)})[v_1]\arrow[Rightarrow]{d}\\ 
\THH(\BP\langle 2\rangle ;k(1))[v_0]\arrow[Rightarrow]{r}  &  \THH(\BP\langle 2\rangle ;\BP\langle 1\rangle)\,. \\ 
\end{tikzcd}
\]
In~\cite{AKCH24}, the authors computed the first Bockstein spectral sequence in each composite of spectral sequences above. 
\begin{theorem}[{\cite[
Theorem~3.8,~Theorem~4.6]{AKCH24}}]
Let $\mathrm{B}\langle 2\rangle$ be an arbitrary $\mathrm{E}_3$-$\mathrm{MU}$-algebra form of $\mathrm{BP}\langle 2\rangle$ at $p>2$ and let $\mathrm{B}\langle 2\rangle:=\tmf_1(3)$ at $p=2$. 
\begin{enumerate}
\item There is an isomorphism 
\[ 
\THH_*(\mathrm{B}\langle 2\rangle ;\mathbb{Z}_{(p)})=\mathbb{F}_p\langle \lambda_1 ,\lambda_2\rangle \otimes \left ( \mathbb{Z}_{(p)}\oplus T_0^2 \right )
\]
where 
\[ T_0^2 = \oplus_{s\ge 1} \mathbb{Z}/p^s\otimes \mathbb{Z}_{(p)}[\mu_3^{p^s}]\otimes \mathbb{Z}_{(p)}\{\lambda_{s+2}\mu_3^{jp^{s-1}} \mid 0\le j\le p-2 \} \,. 
\]
\item There is an isomorphism
\[
\THH_*(\mathrm{B}\langle 2\rangle ;k(1))= \mathbb{F}_p\langle \lambda_1 \rangle \otimes \left ( \mathbb{F}_p[v_1]\oplus T_1^2 \right ) 
\]
where 
\[ T_1^2=\bigoplus_{s\ge 1}\mathbb{F}_p[v_1]/(v_1^{r(s,1)})\otimes \mathbb{F}_p[\mu_3^{p^s}]\otimes \mathbb{F}_p\langle \lambda_{s+2}\rangle \otimes \mathbb{F}_p\{ \lambda_{s+1}\mu_3^{jp^{s-1}} \mid 0\le j\le p-2 \} \,.\]
\end{enumerate}
\end{theorem}

Note that the map 
\[ 
\THH_*(\mathrm{B}\langle 2\rangle , k(1))\longrightarrow \THH_*(\mathrm{B}\langle 2\rangle ,\mathbb{F}_p)
\]
is injective modulo $v_1$. 
\begin{theorem}
The 	
\end{theorem}



From this, we produce a second bound on $\mathrm{THH}_*(\mathrm{B}\langle 2\rangle;\mathrm{B}\langle 1\rangle)$. 
\begin{corollary}
There is an inequality 
\[ |\THH_k(\mathrm{B}\langle 2\rangle |\le \THH_k(\mathrm{B}\langle 2\rangle , k(1))[v_0]\,. \]
In particular, 
\[
\THH_k(\mathrm{B}\langle 2\rangle , \mathrm{B}\langle 1\rangle ) =0 
\]
when $k\not \equiv 0,-1\mod 2p$. 
\end{corollary}
