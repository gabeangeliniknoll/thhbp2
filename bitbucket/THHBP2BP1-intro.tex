% root file is THHBP2.tex

\section{Introduction}
Topological Hochschild homology is a rich invariant of rings, or more generally ring spectra, with applications to such fields as deformation theory, string topology, and integral $p$-adic Hodge theory. It is also a first order approximation to algebraic K-theory in a sense made precise using Goodwillie's calculus of functors, which is our primary motivation. Algebraic K-theory of ring spectra that arise in chromatic stable homotopy theory are of particular interest because of the program of Ausoni-Rognes \cite{AR02} which, in a broad sense, suggests that the arithmetic of structured ring spectra encoded in algebraic K-theory is intimately connected to chromatic complexity. 

One of the most fundamental objects in chromatic stable homotopy theory is the Brown-Peterson spectrum $BP$, which is a complex oriented cohomology theory associated to the universal $p$-typical formal group. The coefficients of $BP$ are a polynomial algebra over $\mathbb{Z}_{(p)}$ on generators $v_i$ for $i\ge 1$, and we may form truncated versions of $BP$, denoted $\tBP{n}$ by coning off a regular sequence $(v_{n+1},v_{n+2}, \ldots )$. 
By convention $\tBP{-1}=H\mathbb{F}_p$ and when $n=0,1$, we can produce models for $\tBP{n}$ by letting $\tBP{0}=H\mathbb{Z}_{(p)}$, and $\tBP{1}=\ell$ where $\ell$ is the Adams summand of complex topological K-theory $ku$. Until recently, the previous list exhausted the examples of $\tBP{n}$ that were known to have models as $E_{\infty}$-ring spectra. However, in the last decade, models for $\B$ as an $E_{\infty}$-ring spectrum were constructed at the prime $p=2$ by Lawson-Naumann \cite{LawsonNaumann} and at the prime $p=3$ by Hill-Lawson \cite{HillLawson}. Lawson-Naumann \cite{LawsonNaumann} use the theory of topological Modular forms with a $\Gamma_1(3)$-structure to construct an $E_{\infty}$ model for $\B$ at the prime $2$ and Hill-Lawson \cite{HillLawson} use the theory of topological automorphic forms associated to a Shimura curve of discriminant $14$ to construct an $E_{\infty}$ model for $\B$ at the prime $p=3$. This is especially interesting in view of recent work of Lawson \cite{Law18} at $p=2$ and Senger \cite{Sen17} for odd primes, where they prove that no model for $\tBP{n}$ as an $E_{\infty}$-ring spectrum exists for $n\ge 4$. %This result was also recently extended to all odd primes by Senger \cite{Sen17}. 

The main theorem of this paper is a computation of topological Hochschild homology of $\B$ with coefficients in $\tBP{1}$ at the prime $3$. %In parallel work to appear, we compute $THH_*(\B;\B/3)$ by a somewhat different approach. 
%In future work, we plan to extend these computations to an integral calculation of $THH_*(\B)$. 
For small values of $n$, the calculations of $THH_*(\tBP{n})$ are known and of fundamental importance. The first known computations of topological Hochschild homology are B\"okstedt's calculations of  $THH_*(\tBP{-1})$ and $THH_*(\tBP{0})$ in \cite{Bok85}. To illustrate how fundamental these computations are, we point out the computation 
\[THH_*(\tBP{-1}) \cong P(\mu_0)\]
where $|\mu_0|=2$, now referred to as B\"okstedt periodicity, is the linchpin for new proof of Bott periodicity \cite{HN19}.

In McClure-Staffledt \cite{McClureStaffeldt}, they compute the Bockstein spectral sequence 
\[ THH_*(\tBP{1}; H\mathbb{F}_p)[v_1]\Rightarrow THH_*(\tBP{1} ; k(1) ).\]
This result is extended by Angeltveit-Hill-Lawson \cite{AHL} where they compute the square of spectral sequences 
\[ 
\xymatrix{
THH_*(\tBP{1}; H\mathbb{F}_p)[v_0,v_1] \ar@{=>}[r] \ar@{=>}[d] & THH_*(\tBP{1}; H\mathbb{Z}_{(p)})_p[v_1] \ar@{=>}[d] \\
THH_*(\tBP{1}; k(1))[v_0] \ar@{=>}[r] & THH_*(\tBP{1};\tBP{1})_p.
}
\]
This gives a complete answer for the ``integral'' calculation $THH_*(\tBP{1})$. 


When $n=2$, the calculation $THH_*(\B;H\F_p)$ follows naturally from \cite{AngeltveitRognes} as we discuss in Section \ref{sec prelim}, but no further results towards $THH_*(\B)$ are known.

In the present paper, we compute the square of spectral sequences
\[ 
\xymatrix{
THH_*(\tBP{2}; H\mathbb{F}_p)[v_0,v_1] \ar@{=>}[r] \ar@{=>}[d] & THH_*(\tBP{2}; H\mathbb{Z}_{p})[v_1] \ar@{=>}[d] \\
THH_*(\tBP{2}; k(1))[v_0] \ar@{=>}[r] & THH_*(\tBP{2};\tBP{1})_p,
}
\]
which is slightly more complex computationally (in a precise sense) than the result of Angeltveit-Hill-Lawson \cite{AHL}, though many of the techniques developed in \cite{AHL} and \cite{McClureStaffeldt} carry over.

We apply a new tool, however, introduced by the first author and Salch, called the topological Hochschild-May spectral sequence \cite{THH-May}. This allows one to compute 
\[THH_*(\tBP{2},\tBP{1})\]
directly. This will not replace the Bockstein spectral sequence, however, because the computational difficulty is elevated. Instead we think of it as computing the diagonal of the square and we compare the diagonal to both paths around the square in order to complete the calculation. Combining  all three ways of computing the output therefore allows us to compute all of the differentials and hidden extensions. 

\gabe{Include statements of main results.}

\subsection{Outline of the strategy}
\gabe{Rewrite this section to reflect current strategy.}
\begin{comment}
Beginning with a calculation of
\[\THH_*(\B;\F_p)\] 
we then compute the Bockstein spectral sequences 
\begin{align}
	\label{v_0BSS}\THH_*(\B;\F_p)[v_0]&\implies \THH_*(\B;\Z_{(p)})^{\wedge}_p\\
	\label{v_1BSS}\THH_*(\B;\F_p)[v_1]&\implies \THH_*(\B;k(1)) \\
	\label{v_0v_1BSS}\THH_*(\B;H\Z_{(p)})[v_1]&\implies \THH_*(\B;\tBP{1})\\
	\label{v_1v_0BSS}\THH_*(\B;k(1))[v_0]&\implies \THH_*(\B;\tBP{1})_p.
\end{align}
The first two Bockstein spectral sequences can be identified with multiplicative Adams spectral spectral sequences 
\begin{align}
	\label{v_0ASS}Ext_{\A_*}^*(\F_p; H_*(THH(\B;\Z_{(p)}))&\implies \THH_*(\B;\Z_{(p)})\\
	\label{v_1ASS}Ext_{\A_*}^*(\F_p;H_*THH(\B;k(1)))&\implies \THH_*(\B;k(1)). 
\end{align}
To see this, note that $H_*THH(\B)$ is free over $H_*\B$ and therefore the input becomes
\[Ext_{E(\otau_i)}^*(\F_p;E(\lambda_1,\lambda_2,\lambda_3)\otimes P(\mu_2))\]
for $i=0,1$. Since $E(\otau_1)$ coacts trivially on $E(\lambda_1,\lambda_2,\lambda_3)\otimes P(\mu_2)$ and 
\[Ext_{E(\otau_1)}^*(\F_p,\F_p)\cong P(v_1)\] 
the spectral sequence \eqref{v_1ASS} can be identified with the Adams spectral sequence at $E_1\cong E_2$-pages. For spectral sequence \eqref{v_0ASS} we must choose a minimal resolution so that their is an identification of  \eqref{v_0BSS} with \eqref{v_0ASS} at $E_1$-pages. Therefore these spectral sequences are each multiplicative. 

The spectral sequence \eqref{v_0v_1BSS} can be identified with the relative Adams spectral sequence 
\begin{align}
	\label{v_0v_1ASS}Ext_{\pi_*(H\Z_{(p}\wedge_{\tBP{1}} H\Z_{(p)})}^*(\Z_{(p)}; THH_*(\B;H\Z_{(p)})) \Rightarrow THH_*(\B;\tBP{1}) 
\end{align}
and the spectral sequence \eqref{v_1v_0BSS} can be identified with the relative Adams spectral sequence 
\begin{align}
	\label{v_1v_0ASS}Ext_{\pi_*(k(1)\wedge_{\tBP{1}} k(1))}^*(k(1)_*; THH_*(\B;k(1))) \Rightarrow THH_*(\B;\tBP{1})_p
\end{align}
and therefore, since $\tBP{1}$ and $H\Z_{(p)}$ are commutative ring spectra and $k(1)$ is an $A_{\infty}$ ring spectrum, these spectral sequences are also multiplicative. To identify the $E_2$-terms of \eqref{v_0v_1ASS} and \eqref{v_0v_1BSS} note that by the K\"unneth spectral sequence there is an isomorphism
\[\pi_*H\Z_{(p}\wedge_{\tBP{1}} H\Z_{(p)}\cong E_{\Z_{(p)}}(\overline{\tau}_1).\] 
Then observe that $E_{\Z}(\otau_1)$ coacts trivially on $THH_*(\B;H\Z_{(p)})$ and 
\[ \Ext_{E_{\Z_{(p)}}(\otau_1)}^*(\Z_{(p)};\Z_{(p)})\cong P_{\Z_{(p)}}(v_1). \]
We see that \eqref{v_0v_1ASS} and \eqref{v_0v_1BSS} are therefore isomorphic on $E_1\cong E_2$-terms. To identify the $E_1$-terms of \eqref{v_1v_0ASS} and \eqref{v_1v_0BSS}, note that 
\[ \pi_*(k(1)\wedge_{\tBP{1}} k(1))\cong P(v_1)\otimes E(\overline{\tau}_0)\]
then apply flat base change 
\[ \Ext_{E(\overline{\tau}_0)\otimes k(1)_*}^*(k(1)_*; THH_*(\B;k(1)))\cong \Ext_{E(\overline{\tau}_0)}^*(\F_p; THH_*(\B;k(1))). \]
to produce an isomorphism of the $E_2$-page with \eqref{v_1v_0ASS}
\[ Ext_{E(\overline{\tau}_0)}^*(\F_p; THH_*(\B;k(1))). \]
Then note that we can take a minimal resolution to produce an isomorphism between the $E_1$-term of \eqref{v_1v_0ASS} and the $E_1$-term of \eqref{v_0v_1BSS}. Thus, the spectral sequences \eqref{v_0v_1BSS} and \eqref{v_1v_0BSS} are also multiplicative. 

As in \cite{AHL}, the topological Hochschild cohomology of $\B$ will play a role in our calculations. Recall from \cite{EKMM}, there there is a universal coefficient spectral sequence (UCSS) of the form
\begin{equation}\label{ucss} 
\Ext_{R_*}^*(M_* , N_*) \Rightarrow \pi_*F_{R}(M,N) 
\end{equation}
when $R$ is a ring spectrum, and $M$ and $N$ are (left) $R$-modules. When $R$ is an $E_{\infty}$-algebra the spectral sequence is a differential graded $R_*$-algebra spectral sequence. 

We will use the Eilenberg-Moore spectral sequence
\[ \Tor^*_{E_*(R)}(E_*(M),E_*(N))\Rightarrow E_*(M\wedge_R N) \]
which exists as long as $E_*R$ is flat as a right $R_*$-module and $E$ and $R$ are $E_{\infty}$-ring spectra and $M$ and and $N$ are $R$-modules. 
\end{comment}

\subsubsection*{Conventions}
Fix $p\in\{2,3\}$ throughout. We will write $H_*(-)$ for homology with $\F_p$ coefficients, or in other words, the functor $\pi_*(H\F_p\wedge -)$. We write $\dot{=}$ to mean that an equality holds up to multiplication by a unit. Specifically, $\tBP{2}$ will denote the $E_{\infty}$-model for the second truncated Brown-Peterson spectrum constructed by \cite{LawsonNaumann} at $p=2$ and \cite{HillLawson} at $p=3$. 
We also note that by coning off $v_2$ on $\B$ we may construct $\tBP{1}$ as an $E_{\infty}$-$\tBP{2}$-algebra by \cite{BSS20} and since the $E_{\infty}$-ring spectrum structure on $\tBP{1}$ is unique after $p$-completion, this is equivalent to the $E_{\infty}$ ring spectrum model constructed in \cite{McClureStaffeldt} after $p$-completion. Similarly, we may construct $H\mathbb{F}_p$ as an $E_{\infty}$-$\tBP{2}$-algebra by \cite{BSS20}. Let $k(n)$ denote an $A_{\infty}$-ring spectrum model for the connective cover of the Morava K-theory spectrum $K(n)$. 

When not otherwise specified, tensor products will be taken over $\mathbb{F}_p$ and $HH_*(A)$ denotes the Hochschild homology of a graded $\mathbb{F}_p$-algebra relative to $\mathbb{F}_p$. We will let $P(x)$, $E(x)$ and $\Gamma(x)$ denote a polynomial algebra, exterior algebra, and divided power algebra over $\mathbb{F}_p$ on a generator $x$. 

The dual Steenrod algebra will be denoted $\A_*$ with coproduct $\Delta\co \A_*\to \A_*\otimes \A_*$. Given a right $\A_*$-comodule $M$, its right coaction will be denoted $\nu\co \A\to \A\otimes M$ where the comodule $M$ is understood from the context. The antipode $\chi\co \A_*\to\A_*$, will not play a role except that we will write $\bar{\xi}_i:=\chi(\xi_i)$ and $\bar{\tau}_i:=\chi(\tau_i)$. 