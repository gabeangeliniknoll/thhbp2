%\documentclass[12pt]{amsart}
%\usepackage[margin=1in]{geometry}
%\usepackage{comment}
%\usepackage{spectralsequences}
%\begin{document}
\DeclareSseqGroup \tower {}{
\class(0,0)
\DoUntilOutOfBoundsThenNMore{10}{
\class(\lastx+2,\lasty+1)
\structline
}
}
\DeclareSseqGroup \mu {}{
\tower(0,0)
\DoUntilOutOfBoundsThenNMore{10}{
\class(\lastx+54,\lasty)
\DoUntilOutOfBoundsThenNMore{10}{
\class(\lastx+2,\lasty+1)
}
}
}
\DeclareSseqGroup \mudfour{}{
\class(0,0)
\DoUntilOutOfBoundsThenNMore{4}{
    \class(\lastx+2,\lasty+1)
    \structline
}
\class(9,0)
\DoUntilOutOfBounds{
    \class(\lastx+2,\lasty+1)
     \d4(\lastx-2,\lasty-1)
}
}
\DeclareSseqGroup \mudeight{}{
\class(0,0)
\DoUntilOutOfBoundsThenNMore{8}{
    \class(\lastx+2,\lasty+1)
    \structline
}
\class(17,0)
\DoUntilOutOfBounds{
    \class(\lastx+2,\lasty+1)
     \d8(\lastx-2,\lasty-1)
}
}
\DeclareSseqGroup \mudtwozero{}{
\class(0,0)
\DoUntilOutOfBoundsThenNMore{22}{
    \class(\lastx+2,\lasty+1)
    \structline
}
\class(41,0)
\DoUntilOutOfBounds{
    \class(\lastx+2,\lasty+1)
     \d{20}(\lastx-2,\lasty-1)
}
}
\DeclareSseqGroup \mudfourzero{}{
\class(0,0)
\DoUntilOutOfBoundsThenNMore{42}{
    \class(\lastx+2,\lasty+1)
    \structline
}
\class(81,0)
\DoUntilOutOfBounds{
    \class(\lastx+2,\lasty+1)
     \d{40}(\lastx-2,\lasty-1)
}
}
\DeclareSseqGroup \mudeighty{}{
\class(0,0)
\DoUntilOutOfBoundsThenNMore{82}{
    \class(\lastx+2,\lasty+1)
    \structline
}
\class(161,0)
\DoUntilOutOfBounds{
    \class(\lastx+2,\lasty+1)
     \d{80}(\lastx-2,\lasty-1)
}
}
\begin{sseqdata}[ name = BSSv_1,classes=fill, xscale = .15, yscale=.3, title = { $v_1$-torsion in the $E_{\infty}$-page of $v_1$-Bockstein Spectral Sequence for $0\le x\le 64$ }, Adams grading, y tick step = 2, x tick step = 8,x range = {0}{64}, y range = {0}{21} ]
%tower on 1
%\tower

%\lambda_1
%tower on  \lambda_1
%\tower(3,0)

%\lambda_2
%\mu
%differential d_4(v_1^k\mu)=\lambda_2v_1^{k+p^2}
\mudfour(7,0)

%\lambda_1\lambda_2
%\lambda_1\mu
%differential d_4(v_1^k\lambda_1\mu)=\lambda_1\lambda_2v_1^{k+p^2}
\mudfour(10,0)

%\lambda_3
%\mu^2
%differential d_8(v_1^k\mu^2)=\lambda_3v_1^{k+p^3}
\mudeight(15,0)

%\lambda_1\lambda_3
%\lambda_1\mu^2
%differential d_8(v_1^k\lambda_1\mu^2)=\lambda_1\lambda_3v_1^{k+p^3}
\mudeight(18,0)

%\lambda_2\lambda_3
%\lambda_3\mu
%differential d_4(v_1^k\lambda_3\mu)=\lambda_3\lambda_2v_1^{k+p^2}
\mudfour(22,0)

%\lambda_2\mu
%\mu^4
%differential d_20(v_1^k\mu^4)=\lambda_2\muv_1^{k+20}
 \mudtwozero(23,0)

%\lambda_1\lambda_2\lambda_3
%\lambda_1\lambda_3\mu
%differential d_4(v_1^k\lambda_1\lambda_3\mu)=\lambda_3\lambda_1\lambda_2v_1^{k+p^2}
\mudfour(25,0)

%\lambda_1\lambda_2\mu
%\lambda_1\mu^4
%differential d_20(v_1^k\lambda_1\mu^4)=\lambda_1\lambda_2\muv_1^{k+20}
\mudtwozero(26,0)

%\lambda_2\lambda_3\mu
%\lambda_2\mu^3
%differential d_8(v_1^k\lambda_2\mu^3)=\lambda_3\lambda_2\muv_1^{k+p^2}
\mudeight(38,0)

%\lambda_2\mu^2
%\mu^3
%differential d_4(v_1^k\mu^3)=\lambda_2\mu^2v_1^{k+p^2}
\mudfour(39,0)

%\lambda_1\lambda_2\lambda_3\mu
%\lambda_1\lambda_2\mu^3
%differential d_8(v_1^k\lambda_1\lambda_2\mu^3)=\lambda_1\lambda_3\lambda_2\muv_1^{k+p^2}
\mudeight(41,0)

%\lambda_1\lambda_2\mu^2
%\lambda_1\mu^3
%differential d_4(v_1^k\lambda_1\mu^3)=\lambda_1\lambda_2\mu^2v_1^{k+p^2}
\mudfour(42,0)

%\lambda_3\mu^2
%\mu^4
%differential d_40(v_1^k\mu^4)=\lambda_3\mu_2^2v_1^{k+p^3}
\mudfourzero(47,0)

%\lambda_1\lambda_3\mu^2
%differential d_40(\lambda_1v_1^k\mu^4)=\lambda_1\lambda_3\mu_2^2v_1^{k+p^3}
\mudfourzero(50,0)

%\lambda_2\lambda_3\mu^2
%\lambda_3mu^3
%differential d_4(v_1^k\lambda_3\mu^3)=\lambda_2\lambda_3\mu^2v_1^{k+p^2}
\mudfour(54,0)

%\lambda_1\lambda_2\lambda_3\mu^2
%\lambda_1\lambda_3\mu^3
%differential d_4(v_1^k\lambda_1\lambda_3\mu^3)=\lambda_1\lambda_2\lambda_3\mu^2v_1^{k+p^2}
\mudfour(57,0)
 
\end{sseqdata}
%\printpage[ name = BSSv_1, page = 4] 
%\newpage
%\printpage[ name = BSSv_1, page = 8]
%\newpage
%\printpage[ name = BSSv_1, page = 20] 
%\newpage
%\printpage[ name = BSSv_1, page = 40] 
%\newpage
\printpage[ name = BSSv_1, page = 41] 


%\end{document}
