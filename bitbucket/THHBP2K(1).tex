\subsection{Topological Hochschild homology of $\tBP{2}$ with $K(1)$ coefficients}

We begin by computing $K(1)_*(\tBP{2})$. This requires determining $\eta_R(v_{2+n})$ in $K(1)_*(BP)$ modulo the ideal generated by $(\eta_R(v_3), \ldots, \eta_R(v_{1+n}))$. We will need the following. 

\gabe{Double check reference}
\begin{lem}\label{lem:rightunit}{\cite[Lemma A.2.2.5]{greenbook}} Let $v_n$ denote the Araki generators. Then there is the following equality in $BP_*(BP)$
\[
\sum_{i,j\geq 0}\hspace{-5pt}\mbox{}^F\: t_i\eta_R(v_j)^{p^i}=\sum_{i,j\geq 0}\hspace{-5pt}\mbox{}^F\:v_it_j^{p^i}
\]
\end{lem}

In this section, the distinction between Hazewinkel generators and Araki generators is unimportant, as the two sets of generators coincide modulo $p$. In $K(1)_*BP$, we have killed all $v_i$'s except $v_1$, which gives us the following equation
\[
\sum_{i,j\geq 0}\hspace{-5pt}\mbox{}^F\: t_i\eta_R(v_j)^{p^i}=\sum_{k\geq 0}\hspace{-4pt}\mbox{}^F\:v_1t_k^{p}
\]
Note that the following degrees of the terms:
\begin{align*}
	|v_1t_j^p|&= 2(p^{j+1}-1)\\
	|t_i\eta_R(v_j)^{p^i}|&= 2(p^{i+j}-1)
\end{align*}
Since we are interested in the term $\eta_R(v_{2+n})$, we collect all the terms on the left of degree $2(p^{2+n}-1)$. Thus we are summing over the ordered pairs $(i,j)$ such that $i+j=2+n$. Since we only care about $\eta_R(v_{2+n})$ modulo $\eta_R(v_3), \ldots, \eta_R(v_{1+n})$ we only need to collect the terms where $j=1, 2$, or $2+n$. This shows that 
\[
t_{1+n}\eta_R(v_1)^{p^{n+1}}+t_n\eta_R(v_2)^{p^n}+\eta_R(v_{n+2})=v_1t_{n+1}^p
\]
The value of $\eta_R$ on $v_1$ and $v_2$ can also be computed by Lemma \ref{lem:rightunit}. One obtains, in $K(1)_*(BP)$, the following
\gabe{Double check these formulas.}
\begin{align*}
\eta_R(v_1)&=v_1\\
\eta_R(v_2)&=v_1t_1^p-t_1v_1^p.
\end{align*}
Combining these observations, we obtain

\begin{lem}
	In $K(1)_*(BP)$, the following congruence is satisfied
	\[
	\eta_R(v_{2+n})\equiv v_1t_{n+1}^p-v_1^{p^n}t_1^{p^{n+1}}t_n+v_1^{p^{n+1}}(t_1^{p^n}t_n-t_{n+1})  \mod(\eta_R(v_3), \ldots, \eta_R(v_{1+n}))
	\]
	for $n\geq 1$.
\end{lem}

Consequently, we have the following corollary. 

\begin{cor}
	There is an isomorphism of $K(1)_*$-algebras
	\[
	K(1)_*(\B)\cong K(1)_*(BP)/(v_1t_{n+1}^p-v_1^{p^n}t_1^{p^{n+1}}t_n+v_1^{p^{n+1}}(t_1^{p^n}t_n-t_{n+1}) \mid n\geq 1)
	\]
\end{cor}

Define elements
\[ u_n:=v_1^{\frac{1-p^n}{p-1}}t_n.\] 
These elements are in degree 0, and therefore there is an isomorphism of $K(1)_*$-algebras
\[
K(1)_*(\B)\cong K(1)_*\otimes_{\F_p}K(1)_0(\B).
\]
The calculations above imply the following corollary.

\begin{cor}
	There is an isomorphism of $\F_p$-algebras
	\[
	K(1)_0 (\B) \cong P(u_i\mid i\geq 1)/(u_{n+1}^p-u_1^{p^{n+1}}u_n+u_1^{p^n}u_n-u_{n+1}\mid n\geq 1).
	\]
\end{cor}

Our goal is to use this and the $K(1)$-based B\"okstedt spectral sequence to compute the $K(1)$-homology of $\THH(\B)$. This is a spectral sequence of the form 
\[
E^2_{s,t}= \HH^{K(1)_*}_s(K(1)_*(\B))\implies K(1)_{s+t}(\THH(\B)).
\]
The above considerations imply
\[
E_{*,*}^2\cong K(1)_*\otimes \HH_*^{\F_p}(K(1)_0\B). 
\]

The following results will be useful for our calculation.

\begin{lem}[\cite{MilneLEC}]\label{lem:etale}
	Let $V = \Spec(A)$ be a nonsingular affine variety over a field $k$. Let $W$ be the subvariety of $V\times \mathbb{A}^n$ defined by equations
	\[
	g_i(Y_1, \ldots, Y_n)=0, \, g_i\in A[Y_1, \ldots, Y_n],\, i=1,\ldots , n.
	\]
	Then the projection map $W\to V$ is \'etale at a point $(P;b_1, \ldots, b_n)$ of $W$ if and only if the Jacobian matrix $\begin{pmatrix}
		\frac{\partial g_i}{\partial Y_j}
	\end{pmatrix}$ is a nonsingular matrix at $(P; b_1, \ldots , b_n)$.
\end{lem} 

\begin{thm}[\'Etale Descent, \cite{WeibelGeller}]\label{etaledescent}
	Let $A\hookrightarrow B$ be an \'etale extension of commutative $k$-algebras. Then there is an isomorphism
	\[
	\HH_*(B)\cong \HH_*(A)\otimes_A B
	\]
\end{thm}

\begin{ex}
	Consider the subalgebra
	\[
	\F_p[u_1,u_2]/(u_2^p-u_1^{p^2+1}+u_1^{p+1}-u_2=f_1).
	\]
	We will regard this as a $\F_p[u_1]$-algebra. The partial derivative $\partial_{u_2}f_1$ is $-1\pmod{p}$, and therefore a unit at every point. Then Lemma \ref{lem:etale} tells us that this algebra is then \'etale over $\F_p[u_1]$.
\end{ex} 

By the same argument given above, we claim that there are a sequence of sub-algebras $A_n$ of 
	\[
	A:= K(1)_0 (\B)\cong\F_p[u_i\mid i\geq 1]/(u_{n+1}^p-u_1^{p^{n+1}}u_n+u_1^{p^n}u_n-u_{n+1}\mid n\geq 1)
	\]
such that each map $A_i\hookrightarrow A_{i+1}$ is an \'etale extension. Here 
\begin{align*}
	A_0:=\mathbb{F}_p[u_1]\\
	A_n:=\F_p[u_1,u_2,\ldots u_n,u_{n+1}]/(u_{k+1}^p-u_1^{p^{k+1}}u_k+u_1^{p^k}u_k-u_{k+1} =f_{k} \mid 1\le k\le n)
\end{align*}
and the partial derivative 
\[\partial_{u_k}f_{k}=-1\pmod{p}\] 
for all $1<k\le n$ and therefore a unit at each point. The claim then follows by Lemma \ref{lem:etale}.

By the \'etale base change formula for Hochschild homology in Theorem \ref{etaledescent}, there is an isomorphism 
\[ \HH_*^{\F_p}(A_{i+1})\cong \HH_*^{\F_p}(A_i)\otimes_{A_i}A_{i+1}\]
and since the functors $\HH_*(-)$ and $ \HH_*^{\F_p}(A_1)\otimes_{A_1}(-)$ commute with filtered colimits of $\F_p$-algebras, there are isomorphisms 
	\[ 
	\begin{array}{rcl} 
		\HH_*^{\F_p}(A) & \cong &\HH_*^{\F_p}(\colim A_n) \\
				         & \cong & \colim  \HH_*^{\F_p}(A_n) \\
				         & \cong & \colim \HH_*^{\F_p}(A_0)\otimes_{A_0}A_n \\
				         & \cong & \HH_*^{\F_p}(A_0)\otimes_{A_0}A. \\
	\end{array}
	\]
Consequently,
\[ \HH_*^{K(1)_*}(K(1)_*(\B))\cong K(1)_*\otimes E(\sigma t_1)\otimes K_0(\B) \]
and therefore, since $\sigma t_1\dot{=}\lambda_1$ mod $p$, 
\[ K(1)_*(\THH(\B))\cong K(1)_*(\B)\otimes E(\lambda_1) \]
and 
\[ \THH_*(\B;K(1))\cong K(1)_*\otimes E(\lambda_1). \]
In other words, 
\[ THH_*(\B;k(1)) \cong F\oplus T\]
where $F$ is a free $P(v_1)$-module generated by $1$ and $\lambda_1$ and $T$ is a torsion $P(v_1)$-module. 
\begin{comment}	
We therefore have the input needed to compute the Eilenberg-Moore spectral sequence,
\[
\Tor^{K(1)_*R}(K(1)_*K(1), K(1)_*\THH(R))\implies K(1)_*\THH(R;K(1))
\]
From the previous computation, the $E^2$-term is concentrated in $\Tor_0$ and is 
\[
K(1)_*K(1)\otimes E(\lambda_1).
\]		         
Thus, every class besides $1$ and $\lambda_1$ is $v_1$-torsion in $\THH_*(R;k(1))$. Since $\THH(R;K(1))$ is a $K(1)$-module, this implies that 
\[
\THH(R;K(1))\simeq K(1)\vee \Sigma^{2p-1}K(1).
\]
\end{comment}

In summary, we have proven the following theorem.
\begin{thm}\label{thm:K(1)coeff} The following hold:
	\begin{enumerate}
		\item The $K(1)$-homology of $\THH(\B;K(1))$ is $K(1)_*K(1)\otimes E(\lambda_1)$
		\item There is a weak equivalence
		\[ K(1)\vee \Sigma^{2p-1}K(1)\simeq \THH(\B;K(1)).\]
		\item The $v_1$-torsion free part of $\THH(\B;k(1))$ is generated by $1$ and $\lambda_1$.
	\end{enumerate}
\end{thm}
